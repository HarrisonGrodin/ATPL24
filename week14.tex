\documentclass[11pt]{article}

\usepackage{ifthen}
\usepackage{xparse}
\usepackage{enumitem}
\usepackage[utf8]{inputenc}
\usepackage{amsmath}
\usepackage{amssymb}
\usepackage{stmaryrd}
\usepackage{amsthm}
\usepackage{mathtools}
\usepackage{proof}
\usepackage{colonequals}
\usepackage{comment}
\usepackage{textcomp}
\usepackage[us]{optional}
\usepackage{color}
\usepackage{url}
\usepackage{verbatim}
\usepackage{graphics}
\usepackage{mathpartir}
\usepackage{tikz}
\usepackage{todonotes}

% these two are used to create the wavy division sign
\usepackage{stackengine}
\usepackage{scalerel}

\usepackage{hyperref}
\usepackage[nameinlink, capitalise]{cleveref}

\usepackage{array}


%% Beamer defines a range of its own theorems
\ifx\beamer\undefined

\theoremstyle{plain}
\newtheorem{theorem}{Theorem}[section]
\newtheorem{lemma}[theorem]{Lemma}
\newtheorem{corollary}[theorem]{Corollary}

\theoremstyle{definition}
\newtheorem*{remark}{Remark}
\newtheorem*{notation}{Notation}
\newtheorem{definition}{Definition}
\newtheorem{conjecture}{Conjecture}
\newtheorem{example}{Example}

\else 
  %% Intentionally left blank
\fi

\makeatletter
\newcommand\xlabel[2][]{\phantomsection\def\@currentlabelname{#1}\label{#2}}
\makeatother

\newenvironment{rules}[1][{}]{\begin{mathpar}}{\end{mathpar}}

%% This puts the name on the top
\NewDocumentCommand{\defrule}{o o m m}{
  \inferrule*[lab=\IfNoValueTF{#1}{}{\textsc{[#1]}}]
  { #3 }
  { #4 }
  \IfNoValueTF{#2}{}{\xlabel[#1]{\ifdefined\InApx{apx:}\else\fi#2}}
}


\NewDocumentCommand{\ruleref}{m}{{[\textsc{\nameref{#1}}]}}


\usepackage[abt]{pfpl-syntax}
\usepackage{pfpl-judgments}

%% \xMapsto command
% \usepackage{mathtools}
\usepackage{stmaryrd}

\makeatletter
\newcommand{\xMapsto}[2][]{\ext@arrow 0599{\Mapstofill@}{#1}{#2}}
\def\Mapstofill@{\arrowfill@{\Mapstochar\Relbar}\Relbar\Rightarrow}
\makeatother

%% Instructor-only remarks. These remarks requires the benefit of the hind sight to understand
%% (or foreshadowing) future content so it doesn't make sense to put it in the file
%% Define \isstudentcopy to generate the student version
\definecolor{iremarkcolor}{rgb}{0.0, 0.0, 0.5}
\NewDocumentEnvironment{iremark}{ +b }
{ \ifthenelse{\isundefined{\isstudentcopy}}{
    \begingroup
    \color{iremarkcolor}
    \begin{remark}
    #1
    \end{remark}
    \endgroup
  }{
  }
}{ }

\newcommand{\inlremark}[1]{\ifthenelse{\isundefined{\isstudentcopy}}{\begingroup\color{iremarkcolor}(#1)\endgroup}{}}
\newcommand{\PFPL}{\textbf{\textsf{PFPL}}}
\makeatletter

\NewDocumentCommand{\fstEx}{s O{\tau_1} O{\tau_2} m}{\IfBooleanTF{#1}{\projEx*<1>[#2][#3]{#4}}{\projEx<1>[#2][#3]{#4}}}
\NewDocumentCommand{\sndEx}{s O{\tau_1} O{\tau_2} m}{\IfBooleanTF{#1}{\projEx*<2>[#2][#3]{#4}}{\projEx<2>[#2][#3]{#4}}}

\NewDocumentCommand{\hatM}{}{\hat{M}}

\NewDocumentCommand{\Iff}{}{\,\mathrm{iff}\,}
\NewDocumentCommand{\limp}{}{\supset}

\NewDocumentCommand{\HT}{O{A}}{\mathsf{HT}_{#1}}


\makeatother


\title{15-791 ATPL \\ Week 14 Notes \\ Program Module}
\author{Shengchao Yang}
\date{\today}

\newcommand*{\judge}[0]{\textbf{judge}}

\begin{document}
\maketitle{}
\section*{Overview}
This week we examine the theory of program modules as the application of dependent type theory with universes.
Besides, we discuss the computational effects that program modules carry. Moreover, 
we study the phase distinct between statics and dynamics of program modules.
Lastly, we look into how to use logical framework briefly.
\section*{Elaboration}
For any full scale programming languages, it is important to distinguish 
the underlying type system, which usually is a combination of statics and dynamics, from its presentation to the programmers.
The procedure that translate the external language (abstract syntax) to the internal language (type theory) is called elaboration.

The internal language is a type theory that has well-defined binding and variable scopes,
which, usually, is not the case for the external language. 
During the elaboration, the following things are done:
\begin{itemize}
    \item Scope Resolution. Variables may have the same name but belong to different scopes.
    Thus, scope resolution specifies the scope of each identifier.
    \item Type Inference. Type inference refers to the deduction of the type of expressions.
    \item Type Classes. Type classes emphasize implicit program construction and
    ad hoc polymorphism\footnote{https://www.cs.cmu.edu/~rwh/papers/mtc/short.pdf}. 
    \item Conversion/Subtyping. Conversion is the process of a forgetful transformation where certain information is ignored.
    For instance, an ORD signature requires two fields : a type \texttt{type t} and a comparison function \texttt{val leq : t * t -> bool}.
    The Integer signature has more content than the ORD signature, but it can be converted to the ORD signature by extracting the relevant parts only.
\end{itemize} 
\section*{Extension Types}
In ML modules, sharing specification is practical. Below is an example.
\begin{verbatim}
    signature ORDER =
    sig
        type t
        val compare : t * t -> order
    end

    signature EQ =
    sig
        type t
        val equal : t * t -> bool
    end

    signature COMB =
    sig
        structure Ord : ORDER
        structure Eq  : EQ
        sharing type Ord.t = Eq.t
    end

    structure Comb :> COMB =
    struct
        structure Ord : ORDER =
        struct
            type t = int
            val compare = Int.compare
        end

        structure Eq : EQ =
        struct
            type t = int
            val equal = (op=)
        end
    end
\end{verbatim} 
In the signature of COMB, we can make sure that \texttt{Ord.t} and \texttt{Eq.t} must be the same type. 
To represent such sharing as a theoretical concept of equality, we introduce extension types as an internalization of judgmental equality.
Below are the relevant rules for the static extent \texttt{STATIC}.
\begin{rules}
    \defrule[formation][formation]{\Gamma \vdash S\ \textbf{sig} \\ \Gamma \vdash M : S}{
        \Gamma \vdash \{S | \text{STATIC} \hookrightarrow M \}\ \textbf{sig}
    }
    \\
    \defrule[intro][intro]{\Gamma \vdash N\ \textbf{sig}\\ \Gamma, \text{STATIC} \vdash M \equiv N : S}{
        \Gamma \vdash \text{in}(N) : \{S | \text{STATIC} \hookrightarrow M \}
    }\\
    \defrule[elim][elim]{\Gamma \vdash N : \{S | \text{STATIC} \hookrightarrow M \}}{
        \Gamma \vdash \text{out}(N) : S
    }
    \qquad
    \defrule[inver][inversion]{\Gamma \vdash N : \{S | \text{STATIC} \hookrightarrow M \}}{
        \Gamma, \text{STATIC} \vdash \text{out}(N) \equiv M: S
    }\\
    \defrule[$\beta$][beta]{
        \Gamma \vdash M : S
    }{
        \Gamma \vdash \text{out}(\text{in}(M)) \equiv M: S
    }\qquad
    \defrule[$\eta$][eta]{
        \Gamma \vdash N : \{S | \text{STATIC} \hookrightarrow M \}
    }{
        \Gamma \vdash \text{in}(\text{out}(N)) \equiv N : \{S | \text{STATIC} \hookrightarrow M \}
    }

\end{rules}
\section*{Logical Framework}
Logical framework is a framework for defining logics (type theories). 
Syntactic logical frameworks present logical systems by generators without relations. In other words, it codifies derivations and thus decidable.
Semantic logical frameworks impose equations (definitional equalities) that may not be decidable.
The semantic logical framework introduced in class is a dependently-typed language with a single universe of judgments (sorts) that involves dependency for types classes and has an extensional equality types with reflection.
We use \judge\ as the objects of class and define G\"{o}del's T as the following:
\begin{align*}
    \textbf{tp} &: \judge \\
    \textbf{el} &: \textbf{tp} \to \judge \\
    \textbf{nat} &: \textbf{tp} \\
    \textbf{zero} &: \textbf{el} (\textbf{nat}) \\
    \textbf{succ} &: \textbf{el} (\textbf{nat}) \to \textbf{el} (\textbf{nat}) \\
    \textbf{arr} &: \textbf{tp} \to \textbf{tp} \to \textbf{tp} \\
    \textbf{lam} &: A, B : \textbf{tp} \to (\textbf{el} A \to \textbf{el} B) \to \textbf{el} (\textbf{arr} A B) \\
    \textbf{app} &: A, B : \textbf{tp} \to \textbf{el} (\textbf{arr} A B) \to \textbf{el} A \to \textbf{el} B \\
\end{align*}
More of the details can be found in Harper(2021)\footnote{https://arxiv.org/pdf/2106.01484}.
\end{document}