\documentclass[11pt]{article}

\usepackage{ifthen}
\usepackage{xparse}
\usepackage{enumitem}
\usepackage[utf8]{inputenc}
\usepackage{amsmath}
\usepackage{amssymb}
\usepackage{stmaryrd}
\usepackage{amsthm}
\usepackage{mathtools}
\usepackage{proof}
\usepackage{colonequals}
\usepackage{comment}
\usepackage{textcomp}
\usepackage[us]{optional}
\usepackage{color}
\usepackage{url}
\usepackage{verbatim}
\usepackage{graphics}
\usepackage{mathpartir}
\usepackage{tikz}
\usepackage{todonotes}

% these two are used to create the wavy division sign
\usepackage{stackengine}
\usepackage{scalerel}

\usepackage{hyperref}
\usepackage[nameinlink, capitalise]{cleveref}

\usepackage{array}


%% Beamer defines a range of its own theorems
\ifx\beamer\undefined

\theoremstyle{plain}
\newtheorem{theorem}{Theorem}[section]
\newtheorem{lemma}[theorem]{Lemma}
\newtheorem{corollary}[theorem]{Corollary}

\theoremstyle{definition}
\newtheorem*{remark}{Remark}
\newtheorem*{notation}{Notation}
\newtheorem{definition}{Definition}
\newtheorem{conjecture}{Conjecture}
\newtheorem{example}{Example}

\else 
  %% Intentionally left blank
\fi

\makeatletter
\newcommand\xlabel[2][]{\phantomsection\def\@currentlabelname{#1}\label{#2}}
\makeatother

\newenvironment{rules}[1][{}]{\begin{mathpar}}{\end{mathpar}}

%% This puts the name on the top
\NewDocumentCommand{\defrule}{o o m m}{
  \inferrule*[lab=\IfNoValueTF{#1}{}{\textsc{[#1]}}]
  { #3 }
  { #4 }
  \IfNoValueTF{#2}{}{\xlabel[#1]{\ifdefined\InApx{apx:}\else\fi#2}}
}


\NewDocumentCommand{\ruleref}{m}{{[\textsc{\nameref{#1}}]}}


\usepackage[abt]{pfpl-syntax}
\usepackage{pfpl-judgments}

%% \xMapsto command
% \usepackage{mathtools}
\usepackage{stmaryrd}

\makeatletter
\newcommand{\xMapsto}[2][]{\ext@arrow 0599{\Mapstofill@}{#1}{#2}}
\def\Mapstofill@{\arrowfill@{\Mapstochar\Relbar}\Relbar\Rightarrow}
\makeatother

%% Instructor-only remarks. These remarks requires the benefit of the hind sight to understand
%% (or foreshadowing) future content so it doesn't make sense to put it in the file
%% Define \isstudentcopy to generate the student version
\definecolor{iremarkcolor}{rgb}{0.0, 0.0, 0.5}
\NewDocumentEnvironment{iremark}{ +b }
{ \ifthenelse{\isundefined{\isstudentcopy}}{
    \begingroup
    \color{iremarkcolor}
    \begin{remark}
    #1
    \end{remark}
    \endgroup
  }{
  }
}{ }

\newcommand{\inlremark}[1]{\ifthenelse{\isundefined{\isstudentcopy}}{\begingroup\color{iremarkcolor}(#1)\endgroup}{}}
\newcommand{\PFPL}{\textbf{\textsf{PFPL}}}
\makeatletter

\NewDocumentCommand{\fstEx}{s O{\tau_1} O{\tau_2} m}{\IfBooleanTF{#1}{\projEx*<1>[#2][#3]{#4}}{\projEx<1>[#2][#3]{#4}}}
\NewDocumentCommand{\sndEx}{s O{\tau_1} O{\tau_2} m}{\IfBooleanTF{#1}{\projEx*<2>[#2][#3]{#4}}{\projEx<2>[#2][#3]{#4}}}

\NewDocumentCommand{\hatM}{}{\hat{M}}

\NewDocumentCommand{\Iff}{}{\,\mathrm{iff}\,}
\NewDocumentCommand{\limp}{}{\supset}

\NewDocumentCommand{\HT}{O{A}}{\mathsf{HT}_{#1}}


\makeatother


\title{15-791 ATPL \\ Week 13 Notes \\ Dependant type theory II}
\author{Yosef Alsuhaibani}
\date{\today}

\newcommand*{\judge}[0]{\textbf{judge}}

\begin{document}
\maketitle{}
\section*{Overview}
In this weeks lectures, we laid the ground work for a pure dependant type theory, and talked more about equality and function extensibility. Later, we discussed an impure dependant type theory, and in the end of it all gone over a "how-to" guide to PL, reviewing the method we used to study different type theories. 

\section*{Dependant type theories}
A few important distinctions from previous theories:
\begin{enumerate}
    \item Contexts are ordered: As later terms' types can be dependent on previous terms, the context is of the form $P_1 : A,\, P_2 : A(P_1), \dots \vdash B : C$
    \item Functions and products are now generalized to be dependant: Instead of the usual product $A \times B$ and a arrow $A \xrightarrow{} B$, they are now $(x : A) \times B(x)$, $(x : A) \xrightarrow{} B(x)$, $\exists (x : A). B$, and $\forall (x : A). B$
    \item Elimination rules for positive types are dependant: an example of this is the $If(\--;\--,\--)$ type
    \end{enumerate}
\section*{Identity \& equality}
Recall from last week the discussion of equality and the reflexive term:
\begin{mathpar}
\inferrule[]{\Gamma \entails M : A \quad \Gamma \entails N : A}{\Gamma \entails \isTp{Id_A (M, N)}}

\inferrule[]{\Gamma \entails M : A}{\Gamma \entails refl_A (M) : Id_A(M, M)}

\inferrule[]{\Gamma \entails M \equiv N : A}{\Gamma \entails refl_A(M) : Id_A(M, M)}
\end{mathpar}
Where we defined this relation as the least reflexive; one might consider to extend equality over functions by utilizing this definition and deriving a rule of the form:
\begin{mathpar}
\inferrule[funext?]{\Gamma, x : A \entails \_ : Id_B(f(x), g(x)) }{\Gamma \entails \_ : Id_{(x : A) \xrightarrow{} B}(f,g)}
\end{mathpar}
Unfortunately, this is not derivable; a workaround to get  function extensionality is to postulate/add an axiom that is exactly this rule into the language, and so we introduce a new term:
\begin{mathpar}
\inferrule[funext]{\Gamma, x : A \entails \_ : Id_B(f(x), g(x)) }{\Gamma \entails FUNEXT(f,g): (x : A) \xrightarrow{} Id_B(f(x),g(x))}
\end{mathpar}
However there is still a problem: we can't quite do anything with this  $J(FUNEXT(f,g), x.R)$! this term is a function, and we are back to square one. There are a few alternatives:
\begin{enumerate}
    \item Extensional equality: admitting that $Id_{\star}(\--,\--)$ is the ``wrong animal" for the job
    \item Observational type theory: Equality should be dependant on the type!
    \item Cubical type theory: Equality via reasoning about ``paths"
\end{enumerate}
\newpage
\section*{Extensional equality}
Changing the $Id_A$ primitive:
\begin{mathpar}
\inferrule[]{\isTp{A} \quad M : A \quad N : A }{\isTp{Eq_A(M,N)}}
\qquad
\inferrule[]{M \equiv N : A }{\star : Eq_A(M,N)}
\qquad
\inferrule[]{P : Eq_A(M,N) \quad Q : Eq_A(M,N)}{P \equiv Q : Eq_A(M,N)}
\end{mathpar}
This allows us to derive $Eq_{A \xrightarrow{} B}(f,g)$ as $(x : A) \xrightarrow{} (y : A) \xrightarrow{} Eq_A(x,y) \xrightarrow{} Eq_B(f(x),g(x)) $
\section*{Observational type theory}
Instead of trying to define equality, we can specialize over the type $A$, such that our equality is inductive on the structure of $A$; we didn't discuss more on this topic in class, instead we moved on to the third alternative.
\section*{Cubical type theory}
But first, a quick intro to universes
\subsection*{Universes}
The main goal is to develop a notion of ``computing" with ``types as data", for example:
\begin{enumerate}
    \item ``Polymorphism": where for a term $\Lambda A. \lambda (x : A). x : (?)$, what would the type of this expression be?
    \item ``abstract types"(modules): where say for a module $[Nat,5] : (?)$
\end{enumerate}
Universes is exactly the notion we are looking for; an early formulation from ML in '71 is the following:
\begin{mathpar}
\inferrule[]{.}{\isTp{\mathcal{U}}}{}
\qquad
\inferrule[]{\isTp{A}}{A : \mathcal{U}}{}
\qquad
\inferrule[]{A : \mathcal{U}}{\isTp{A}}{}
\end{mathpar}
However it was possible to show that $\mathcal{U} : \mathcal{U}$ led to some inconsistency, and Girard proved it so; so in '73 a new revision came in that disallowed exactly that by introducing ``levels":
\begin{mathpar}
\inferrule[]{.}{\isTp{\mathcal{U_i}}_{i + 1}}{}
\qquad
\inferrule[]{\isTp{A}_i}{A : \mathcal{U}_i}{}
\qquad
\inferrule[]{A : \mathcal{U}_i}{\isTp{A}_i}{}
\end{mathpar}
Where the levels $i \geq 0$, and that each level $i$ is closed by bool, $Eq_A$, $x$,$\xrightarrow{}$, but not $U_i$ or any level higher. Some additional rules 
\begin{mathpar}
\inferrule[]{\isTp{A}_i}{\isTp{A}_{i + 1}}{}
\quad
\inferrule[]{\isTp{A \equiv B}_i}{\isTp{A \equiv B}_{i + 1}}{}
\end{mathpar}
\subsection*{Univalence axiom}
Now that we have universes in our type theory, the two notions of equivalnace/equality $Id_A$ is isomorphic to $Eq_A$, as we can introduce a new notion $Path_{A}(M,N)$ i.e evidence that $M$ can be ``deformed" to $N$; $Path_M(A,B)$ is defined as $Eq(A,B)$!
\\
\\
Now tht we hvae these paths, we want to reason with them as so:
\begin{mathpar}
\inferrule[]{M \, {p \over{}}\, N : A}{Path(p) : Path_A(M,N)}{}
\end{mathpar}
And these paths form "shapes" where we we can have path connecting between two paths.
\newpage
\section*{Adding computations}
With the pure dependant type theory, we can extend it with computations by adding $\Gamma \vdash X \; \mathrm{Ctype}$, $\Gamma \vdash X \equiv Y$, $\Gamma \vdash C : X$, $\Gamma \vdash C \equiv C' : X$. With this new addition, we might want to study this new type theory, so it might be useful to review our methodology:
\begin{enumerate}
    \item Start with the operational semantics of the langauge, the transitions $M \xrightarrow{} M'$ and $M \; \mathrm{Val}$
    \item Create a logic of programs that specifies behavior via logical relations
    \item Validate the theory with theorems $\Gamma \vdash \isTp{A}$ and $\Gamma >> A \equiv A'$
    \item Account for effects
    \item Add phases similar to Kripke's logical relations
\end{enumerate}
A type system consists of
\begin{enumerate}
    \item A p.e.r with symmetry and transitivity such that $A = B$ is the exact equality of types.
    \item For $\isTp{A}$ we have $M \doteq N \in A$ a p.e.r such that if $A \doteq B$ then $\_ \doteq \_ \in A = \_ \doteq \_ \in B$
    \item Closed under head expansion so
    \begin{itemize}
        \item If $A' \xrightarrow{} A$ and $A \doteq B$ then $A' \doteq B$
        \item If $M' \xrightarrow{} N$ and $M = N \in AA$ then $M' = N \in A$
    \end{itemize}
    \item Define $\Gamma \doteq \Gamma'$ and $\gamma \doteq \gamma'$
    \item Define $\Gamma >> A \doteq B$ presupposing $\Gamma \doteq \Gamma'$ iff $\gamma \doteq \gamma' \in \Gamma$ implies $\hat{\gamma}A \doteq \hat{\gamma}B$
    \item FTLR: if something was derivable then it is true which is the underlying mechanism behind the ``Canonical forms" lemma, for example:
\end{enumerate}
\end{document}