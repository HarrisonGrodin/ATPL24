% latexmk -pdf week3

\documentclass[letterpaper]{article}

\usepackage{ifthen}
\usepackage{xparse}
\usepackage{enumitem}
\usepackage[utf8]{inputenc}
\usepackage{amsmath}
\usepackage{amssymb}
\usepackage{stmaryrd}
\usepackage{amsthm}
\usepackage{mathtools}
\usepackage{proof}
\usepackage{colonequals}
\usepackage{comment}
\usepackage{textcomp}
\usepackage[us]{optional}
\usepackage{color}
\usepackage{url}
\usepackage{verbatim}
\usepackage{graphics}
\usepackage{mathpartir}
\usepackage{tikz}
\usepackage{todonotes}

% these two are used to create the wavy division sign
\usepackage{stackengine}
\usepackage{scalerel}

\usepackage{hyperref}
\usepackage[nameinlink, capitalise]{cleveref}

\usepackage{array}


%% Beamer defines a range of its own theorems
\ifx\beamer\undefined

\theoremstyle{plain}
\newtheorem{theorem}{Theorem}[section]
\newtheorem{lemma}[theorem]{Lemma}
\newtheorem{corollary}[theorem]{Corollary}

\theoremstyle{definition}
\newtheorem*{remark}{Remark}
\newtheorem*{notation}{Notation}
\newtheorem{definition}{Definition}
\newtheorem{conjecture}{Conjecture}
\newtheorem{example}{Example}

\else 
  %% Intentionally left blank
\fi

\makeatletter
\newcommand\xlabel[2][]{\phantomsection\def\@currentlabelname{#1}\label{#2}}
\makeatother

\newenvironment{rules}[1][{}]{\begin{mathpar}}{\end{mathpar}}

%% This puts the name on the top
\NewDocumentCommand{\defrule}{o o m m}{
  \inferrule*[lab=\IfNoValueTF{#1}{}{\textsc{[#1]}}]
  { #3 }
  { #4 }
  \IfNoValueTF{#2}{}{\xlabel[#1]{\ifdefined\InApx{apx:}\else\fi#2}}
}


\NewDocumentCommand{\ruleref}{m}{{[\textsc{\nameref{#1}}]}}


\usepackage[abt]{pl-syntax}
\usepackage{pl-judgments}

%% \xMapsto command
% \usepackage{mathtools}
\usepackage{stmaryrd}

\makeatletter
\newcommand{\xMapsto}[2][]{\ext@arrow 0599{\Mapstofill@}{#1}{#2}}
\def\Mapstofill@{\arrowfill@{\Mapstochar\Relbar}\Relbar\Rightarrow}
\makeatother

%% Instructor-only remarks. These remarks requires the benefit of the hind sight to understand
%% (or foreshadowing) future content so it doesn't make sense to put it in the file
%% Define \isstudentcopy to generate the student version
\definecolor{iremarkcolor}{rgb}{0.0, 0.0, 0.5}
\NewDocumentEnvironment{iremark}{ +b }
{ \ifthenelse{\isundefined{\isstudentcopy}}{
    \begingroup
    \color{iremarkcolor}
    \begin{remark}
    #1
    \end{remark}
    \endgroup
  }{
  }
}{ }

\newcommand{\inlremark}[1]{\ifthenelse{\isundefined{\isstudentcopy}}{\begingroup\color{iremarkcolor}(#1)\endgroup}{}}
\newcommand{\PFPL}{\textbf{\textsf{PFPL}}}

\makeatletter

\NewDocumentCommand{\fstEx}{s O{\tau_1} O{\tau_2} m}{\IfBooleanTF{#1}{\projEx*<1>[#2][#3]{#4}}{\projEx<1>[#2][#3]{#4}}}
\NewDocumentCommand{\sndEx}{s O{\tau_1} O{\tau_2} m}{\IfBooleanTF{#1}{\projEx*<2>[#2][#3]{#4}}{\projEx<2>[#2][#3]{#4}}}

\NewDocumentCommand{\hatM}{}{\hat{M}}

\NewDocumentCommand{\Iff}{}{\,\mathrm{iff}\,}
\NewDocumentCommand{\limp}{}{\supset}

\NewDocumentCommand{\HT}{O{A}}{\mathsf{HT}_{#1}}


\makeatother



\title{15-791 ATPL \\ Week 3 Notes}
\author{Nathan Glover and Chase Norman}
\date{\today}
\begin{document}
\maketitle

\section{Exact Equality}

Previously, we defined the unary relation on terms $HT_A(M)$, also written as $M \in A$, denoting that $M$ satisfies the behavioral specification of hereditary termination given by $A$.
Along these lines, we will be defining a binary relation that can be thought of as $EQ_A(M, M')$, which we write $M \dot{=} M' \in A$, representing Exact (or Semantic) Equality in type $A$.
Again, we define this relation by induction on the structure of the closed type $A$. We redefine $M \in A$ to mean $M \dot{=} M \in A$. 
We are encouraged to view $A$ as a specification, making this construction an invocation of the ``propositions as types'' principle. 
Our ``theory of truth'' in defining the satisfaction of the proposition $M \dot{=} M' \in A$ will be based in the computation of $M$ and $M'$. 

\begin{definition}[Exact Equality]
    $$
  \begin{array}{cll}
      M \dot{=} M' \in \unitTy* & \Iff & M, M' \mapsto^* \unitEx* \\
      M \dot{=} M' \in \ansTy* & \Iff & M, M' \mapsto^* \yesEx* \text{ or } M, M' \mapsto^* \noEx* \\
      M \dot{=} M' \in \prodTy*{A_1}{A_2} & \Iff &
         M \mapsto^* \pairEx*{M_1}{M_2} \text{ and } M' \mapsto^* \pairEx*{M_1'}{M_2'}\\ & & \text{ where } M_1 \dot{=} M_1' \in A_1 \text{ and } M_2 \dot{=} M_2' \in A_2 \\
      M \dot{=} M' \in \arrTy*{A_1}{A_2} & \Iff & M \mapsto^* \lamEx*{x}{M_2} \text{ and } M' \mapsto^* \lamEx*{x}{M_2'} \text{ where } M_1 \dot{=} M_1' \in A_1 \\ & & 
         \text{ implies } \Sub{M_1}{x}{M_2} \dot{=} \Sub{M_1'}{x}{M_2'} \in A_2 \\
  \end{array}
  $$
\end{definition}

These definitions essentially derive from our treatment of types as specifications. For each type, we reduce $M$ and $M'$ to values, and close under head expansion for the subterms. 
Note that our definition of exact equaltiy in the arrow type does not require $M_1$ and $M_1'$ to be identical, only semantically equal. 
The impact of this will become apparent as more concepts are added.

Now, we extend to open terms:


\begin{definition}[$\Gamma \gg M \dot{=} M' \in A$]
    $\gamma \dot{=} \gamma' \in \Gamma$ implies $\hat{\gamma} M \dot{=} \hat{\gamma'} M' \in A$ 
\end{definition}

Where $\gamma \dot{=} \gamma' \in \Gamma$ is variable-by-variable equality of the terms in substitutions $\gamma$ and $\gamma'$. 
Shortly, we will define judgemental equality $\equiv$ and prove that $\Gamma \entails{M \equiv N : A}$ implies $\Gamma \gg M \dot{=} N \in A$.
In essence, this is showing that proof implies truth. First, we establish some lemmas about exact equality.

\begin{lemma}[Head Expansion]\label{lem:headexpansion}
    If $M \dot{=} M' \in A$ and $N \mapsto^* M$ then $N \dot{=} M' \in A$. Likewise, if $N' \mapsto^* M'$ then $M \dot{=} N' \in A$. 
\end{lemma}
\begin{proof}
By induction on $A$, direct from the definition of exact equality.
\end{proof}

\begin{lemma}[Partial Equivalence Relation -- P.E.R.]
    For all $A$, exact equality in $A$ is symmetric and transitive. 
\end{lemma}
\begin{proof}
We prove these properties simultaneously, and by induction on $A$. We will require symmetry to prove transitivity in the case of the arrow type. 

\begin{enumerate}
    \item Case $A = \unitTy*$: Direct from the symmetry and transitivity of logical and.
    \item Case $A = \ansTy*$: Symmetry follows from the symmetry of logical and. For transitivity we observe that $M \mapsto^* \yesEx*$ and $M \mapsto^* \noEx*$
    are mutually exclusive for any term $M$. Therefore, if $M \dot{=} M' \in \ansTy*$ and $M' \dot{=} M'' \in \ansTy*$, we know $M$ and $M''$ must evaluate to
    the same value as $M$ and $M''$, completing the proof.
    \item Case $A = \prodTy*{A_1}{A_2}$: 
    \begin{enumerate}
        \item Apply symmetry from the inductive hypothesis on $M_1 \dot{=} M_1' \in A_1$ and $M_2 \dot{=} M_2' \in A_2$. The result follows directly.
        \item Apply transitivity from the inductive hypothesis on $M_1 \dot{=} M_1' \in A_1$ with $M_1' \dot{=} M_1''$ and $M_2 \dot{=} M_2'$ with $M_2' \dot{=} M_2''$.
        The result follows directly.
    \end{enumerate}
    \item Case $A = \arrTy*{A_1}{A_2}$
    \begin{enumerate}
        \item We want to show $M' \dot{=} M \in A$, so assume $M_1' \dot{=} M_1 \in A_1$ and let $M \mapsto^* \lamEx*{x}{M_2}$ and $M' \mapsto^* \lamEx*{x}{M_2'}$. 
        By the inductive hypothesis, we have $M_1 \dot{=} M_1' \in A_1$. With this and the given $M \dot{=} M' \in A$,
        we have $\Sub{M_1}{x}{M_2} \dot{=} \Sub{M_1'}{x}{M_2'} \in A_2$. Applying the inductive hypothesis on this equality completes the proof. 
        \item Using the P.E.R. trick, discussed below.
    \end{enumerate}
\end{enumerate}

To complete the transitivity case for $A = \arrTy*{A_1}{A_2}$, suppose $M \dot{=} M' \dot{=} M'' \in A$ and we want to show $M \dot{=} M'' \in A$. 
We have $M \mapsto^* \lamEx*{x}{M_2}$, $M' \mapsto^* \lamEx*{x}{M_2'}$, and $M'' \mapsto^* \lamEx*{x}{M_2''}$. Suppose we have inputs $M_1 \dot{=} M_1'' \in A_1$. 
We must show $\Sub{M_1}{x}{M_2} \dot{=} \Sub{M_1''}{x}{M_2''} \in A_2$.

The natural approach would be to input $M_1 \dot{=} M_1'' \in A_1$ into the equalities from $M \dot{=} M' \dot{=} M'' \in A$, obtaining 
$\Sub{M_1}{x}{M_2} \dot{=} \Sub{M_1''}{x}{M_2'} \in A_2$ and $\Sub{M_1}{x}{M_2'} \dot{=} \Sub{M_1''}{x}{M_2''} \in A_2$. However, when we go to apply
transitivity on these, we find that the terms $\Sub{M_1''}{x}{M_2'}$ and $\Sub{M_1}{x}{M_2'}$ do not quite match. 

To remedy this, we require what we dub the ``P.E.R. trick.'' By symmetry, $M_1 \dot{=} M_1'' \in A_1$ implies $M_1'' \dot{=} M_1 \in A_1$.
By transitivity, we obtain that $M_1 = M_1 \in A_1$. This is the trick; any element that is related to another in a P.E.R. is also related to itself.

We then can apply $M_1 = M_1 \in A_1$ on $M \dot{=} M' \in A$ and $M_1 \dot{=} M_1'' \in A_1$ on $M' \dot{=} M'' \in A$ to obtain
$\Sub{M_1}{x}{M_2} \dot{=} \Sub{M_1}{x}{M_2'} \in A_2$ and $\Sub{M_1}{x}{M_2'} \dot{=} \Sub{M_1''}{x}{M_2''} \in A_2$. The two sides match now, 
and transitivity from the inductive hypothesis completes the proof.
\footnote{Analogously, we could have used the P.E.R. trick to find $M_1'' \dot{=} M_1'' \in A_1$ and applied it to $M' \dot{=} M'' \in A$.}
The P.E.R. trick allows us to bridge the gap between these two equalities, placing the transition from $M_1$ to $M_1''$ entirely on one side.  

\end{proof}

Reflexivity will hold if and only if a term is well-typed.

\begin{theorem}[Reflexivity]
    If $\Gamma \entails{\isOfTp{M}{A}}$ then $\Gamma \gg M \in A$, meaning $\Gamma \gg M \dot{=} M \in A$.
\end{theorem}
The proof of this was left as an exercise. It is necessary for this theorem to hold, otherwise the reflexivity inference rule for judgemental equality (Rule \ruleref{R}) is unsound.

\section{Judgemental Equality}\label{sec:jeq}

We define $\Gamma \entails{\isOfTp{M \equiv N}{A}}$ for closed terms $M$ and $N$ via inference rules. 
With the first set of rules, we axiomatically prescribe judgemental equality to be a P.E.R. with reflexivity
for all well-typed terms. 

\begin{mathpar}

\defrule[R][jeq:r]
{\Gamma \entails{\isOfTp{M}{A}}}
{ \Gamma \entails{\isOfTp{M \equiv M}{A}} }

\defrule[S][jeq:s]
{\Gamma \entails{\isOfTp{M \equiv N}{A}} }
{ \Gamma \entails{\isOfTp{N \equiv M}{A}} }

\defrule[T][jeq:t]
{\Gamma \entails{\isOfTp{M \equiv N}{A}} \\ \Gamma \entails{\isOfTp{N \equiv P}{A}} }
{ \Gamma \entails{\isOfTp{M \equiv P}{A}} }

\end{mathpar}

The second set of inference rules describe how to construct equivalent terms from equivalent subterms.

\begin{mathpar}

\defrule[Pair][jeq:pair]
{\Gamma \entails{\isOfTp{M_1 \equiv N_1}{A_1}} \\ \Gamma \entails{\isOfTp{M_2 \equiv N_2}{A_2}}}
{ \Gamma \entails{\isOfTp{\pairEx*{M_1}{M_2} \equiv \pairEx*{N_1}{N_2}}{\prodTy*{A_1}{A_2}}} }

\defrule[Proj][jeq:proj]
{\Gamma \entails{\isOfTp{M \equiv N}{\prodTy*{A_1}{A_2}}} }
{ \Gamma \entails{\isOfTp{M \cdot i \equiv N \cdot i}{A_i}}}

\defrule[Lam][jeq:lam]
{\Gamma, \isOfTp{x}{A_1} \entails{\isOfTp{M \equiv N}{A_2}} }
{ \Gamma \entails{\isOfTp{\lamEx*{x}{M} \equiv \lamEx*{x}{N}}{\arrTy*{A_1}{A_2}}}}

\defrule[App][jeq:app]
{\Gamma \entails{\isOfTp{M \equiv N}{\arrTy*{A_1}{A_2}}} \\ \Gamma \entails{\isOfTp{M_1 \equiv N_1}{A_1}} }
{ \Gamma \entails{\isOfTp{\appEx*{M}{M_1} \equiv \appEx*{N}{N_1}}{A_2}}}

\end{mathpar}

And finally, the third set of rules concern $\beta$ and $\eta$ equality.

\begin{mathpar}

\defrule[$\eta_{\unitTy*}$][jeq:eta1]
{\Gamma \entails{\isOfTp{M}{\unitTy*}}}
{\Gamma \entails{\isOfTp{M \equiv \unitEx*}{\unitTy*}}}

\defrule[$\beta_{\prodTy*{}{}}$][jeq:betaprod]
{\Gamma \entails{\isOfTp{M_1}{A_1}} \\ \Gamma \entails{\isOfTp{M_2}{A_2}}}
{\Gamma \entails{\isOfTp{\pairEx*{M_1}{M_2} \cdot i \equiv M_i}{A_i}}}

\defrule[$\eta_{\prodTy*{}{}}$][jeq:etaprod]
{\Gamma \entails{\isOfTp{M}{\prodTy*{A_1}{A_2}}}}
{\Gamma \entails{\isOfTp{M \equiv \pairEx*{M \cdot 1}{M \cdot 2}}{\prodTy*{A_1}{A_2}}}}

\defrule[$\beta_{\arrTy*{}{}}$][jeq:betaarr]
{\Gamma, \isOfTp{x}{A_1} \entails{\isOfTp{M_2}{A_2}} \\ \Gamma \entails{\isOfTp{M_1}{A_1}}}
{\Gamma \entails{\isOfTp{\appEx*{\lamEx*{x}{M_2}}{M_1} \equiv \Sub{M_1}{x}{M_2}}{A_2}}}

\defrule[$\eta_{\arrTy*{}{}}$][jeq:etaarr]
{\Gamma \entails{\isOfTp{M}{\arrTy*{A_1}{A_2}}}}
{\Gamma \entails{\isOfTp{\lamEx*{x}{\appEx*{M}{x}}}{\arrTy*{A_1}{A_2}}}}

\end{mathpar}

\begin{theorem}[Soundness]
    If $\Gamma \entails{\isOfTp{M \equiv N}{A}}$ then $\Gamma \gg M \dot{=} N \in A$.
\end{theorem}

The proof of this theorem is assigned for homework. The proof should proceed by induction on derivations, and requires the use of reflexivity.
For the cases in the first batch of inference rules, we require the symmetry and transitivity of $\Gamma \gg M \dot{=} N \in A$, which uses the P.E.R. trick.
For the $\beta_{\arrTy*{}{}}$ case, we will need head expansion (Lemma \ref{lem:headexpansion}). 

\end{document}