% latexmk -pdf week3

\documentclass[letterpaper]{article}

\usepackage{ifthen}
\usepackage{xparse}
\usepackage{enumitem}
\usepackage[utf8]{inputenc}
\usepackage{amsmath}
\usepackage{amssymb}
\usepackage{stmaryrd}
\usepackage{amsthm}
\usepackage{mathtools}
\usepackage{proof}
\usepackage{colonequals}
\usepackage{comment}
\usepackage{textcomp}
\usepackage[us]{optional}
\usepackage{color}
\usepackage{url}
\usepackage{verbatim}
\usepackage{graphics}
\usepackage{mathpartir}
\usepackage{tikz}
\usepackage{todonotes}

% these two are used to create the wavy division sign
\usepackage{stackengine}
\usepackage{scalerel}

\usepackage{hyperref}
\usepackage[nameinlink, capitalise]{cleveref}

\usepackage{array}


%% Beamer defines a range of its own theorems
\ifx\beamer\undefined

\theoremstyle{plain}
\newtheorem{theorem}{Theorem}[section]
\newtheorem{lemma}[theorem]{Lemma}
\newtheorem{corollary}[theorem]{Corollary}

\theoremstyle{definition}
\newtheorem*{remark}{Remark}
\newtheorem*{notation}{Notation}
\newtheorem{definition}{Definition}
\newtheorem{conjecture}{Conjecture}
\newtheorem{example}{Example}

\else 
  %% Intentionally left blank
\fi

\makeatletter
\newcommand\xlabel[2][]{\phantomsection\def\@currentlabelname{#1}\label{#2}}
\makeatother

\newenvironment{rules}[1][{}]{\begin{mathpar}}{\end{mathpar}}

%% This puts the name on the top
\NewDocumentCommand{\defrule}{o o m m}{
  \inferrule*[lab=\IfNoValueTF{#1}{}{\textsc{[#1]}}]
  { #3 }
  { #4 }
  \IfNoValueTF{#2}{}{\xlabel[#1]{\ifdefined\InApx{apx:}\else\fi#2}}
}


\NewDocumentCommand{\ruleref}{m}{{[\textsc{\nameref{#1}}]}}


\usepackage[abt]{pl-syntax}
\usepackage{pl-judgments}

%% \xMapsto command
% \usepackage{mathtools}
\usepackage{stmaryrd}

\makeatletter
\newcommand{\xMapsto}[2][]{\ext@arrow 0599{\Mapstofill@}{#1}{#2}}
\def\Mapstofill@{\arrowfill@{\Mapstochar\Relbar}\Relbar\Rightarrow}
\makeatother

%% Instructor-only remarks. These remarks requires the benefit of the hind sight to understand
%% (or foreshadowing) future content so it doesn't make sense to put it in the file
%% Define \isstudentcopy to generate the student version
\definecolor{iremarkcolor}{rgb}{0.0, 0.0, 0.5}
\NewDocumentEnvironment{iremark}{ +b }
{ \ifthenelse{\isundefined{\isstudentcopy}}{
    \begingroup
    \color{iremarkcolor}
    \begin{remark}
    #1
    \end{remark}
    \endgroup
  }{
  }
}{ }

\newcommand{\inlremark}[1]{\ifthenelse{\isundefined{\isstudentcopy}}{\begingroup\color{iremarkcolor}(#1)\endgroup}{}}
\newcommand{\PFPL}{\textbf{\textsf{PFPL}}}

\makeatletter

\NewDocumentCommand{\fstEx}{s O{\tau_1} O{\tau_2} m}{\IfBooleanTF{#1}{\projEx*<1>[#2][#3]{#4}}{\projEx<1>[#2][#3]{#4}}}
\NewDocumentCommand{\sndEx}{s O{\tau_1} O{\tau_2} m}{\IfBooleanTF{#1}{\projEx*<2>[#2][#3]{#4}}{\projEx<2>[#2][#3]{#4}}}

\NewDocumentCommand{\hatM}{}{\hat{M}}

\NewDocumentCommand{\Iff}{}{\,\mathrm{iff}\,}
\NewDocumentCommand{\limp}{}{\supset}

\NewDocumentCommand{\HT}{O{A}}{\mathsf{HT}_{#1}}


\makeatother



\title{15-791 ATPL \\ Week 3 Notes}
\author{Nathan Glover and Chase Norman}
\date{\today}
\begin{document}
\maketitle

\section{Exact Equality}

Previously, we defined the unary relation on terms $HT_A(M)$, also written as $M \in A$, denoting that $M$ satisfies the behavioral specification of hereditary termination given by $A$.
Along these lines, we will be defining a binary relation that can be thought of as $EQ_A(M, M')$, which we write $M \dot{=} M' \in A$, representing Exact (or Semantic) Equality in type $A$.
Again, we define this relation by induction on the structure of the closed type $A$. We redefine $M \in A$ to mean $M \dot{=} M \in A$. 
We are encouraged to view $A$ as a specification, making this construction an invocation of the ``propositions as types'' principle. 
Our ``theory of truth'' in defining the satisfaction of the proposition $M \dot{=} M' \in A$ will be based in the computation of $M$ and $M'$. 

\begin{definition}[Exact Equality]
    $$
  \begin{array}{cll}
      M \dot{=} M' \in \unitTy* & \Iff & M, M' \evalsTo \unitEx* \\
      M \dot{=} M' \in \ansTy* & \Iff & M, M' \evalsTo \yesEx* \text{ or } M, M' \evalsTo \noEx* \\
      M \dot{=} M' \in \prodTy*{A_1}{A_2} & \Iff &
         M \evalsTo \pairEx*{M_1}{M_2} \text{ and } M' \evalsTo \pairEx*{M_1'}{M_2'}\\ & & \text{ where } M_1 \dot{=} M_1' \in A_1 \text{ and } M_2 \dot{=} M_2' \in A_2 \\
      M \dot{=} M' \in \arrTy*{A_1}{A_2} & \Iff & M \evalsTo \lamEx*{x}{M_2} \text{ and } M' \evalsTo \lamEx*{x}{M_2'} \text{ where } M_1 \dot{=} M_1' \in A_1 \\ & & 
         \text{ implies } \Sub{M_1}{x}{M_2} \dot{=} \Sub{M_1'}{x}{M_2'} \in A_2 \\
  \end{array}
  $$
\end{definition}

These definitions essentially derive from our treatment of types as specifications. For each type, we reduce $M$ and $M'$ to values, and close under head expansion for the subterms. 
Note that our definition of exact equaltiy in the arrow type does not require $M_1$ and $M_1'$ to be identical, only semantically equal. 
The impact of this will become apparent as more concepts are added.

Now, we extend to open terms:


\begin{definition}[$\Gamma \gg M \dot{=} M' \in A$]
    $\gamma \dot{=} \gamma' \in \Gamma$ implies $\hat{\gamma} M \dot{=} \hat{\gamma'} M' \in A$ 
\end{definition}

Where $\gamma \dot{=} \gamma' \in \Gamma$ is variable-by-variable equality of the terms in substitutions $\gamma$ and $\gamma'$. 
Shortly, we will define judgemental equality $\equiv$ and prove that $\Gamma \entails{M \equiv N : A}$ implies $\Gamma \gg M \dot{=} N \in A$.
In essence, this is showing that proof implies truth. First, we establish some lemmas about exact equality.

\begin{lemma}[Head Expansion]\label{lem:headexpansion}
    If $M \dot{=} M' \in A$ and $N \to_\beta M$ then $N \dot{=} M' \in A$. Likewise, if $N' \to_\beta M'$ then $M \dot{=} N' \in A$. 
\end{lemma}
\begin{proof}
By induction on $A$, direct from the definition of exact equality.
\end{proof}

\end{document}