\documentclass[letterpaper]{article}

\usepackage{ifthen}
\usepackage{xparse}
\usepackage{enumitem}
\usepackage[utf8]{inputenc}
\usepackage{amsmath}
\usepackage{amssymb}
\usepackage{stmaryrd}
\usepackage{amsthm}
\usepackage{mathtools}
\usepackage{proof}
\usepackage{colonequals}
\usepackage{comment}
\usepackage{textcomp}
\usepackage[us]{optional}
\usepackage{color}
\usepackage{url}
\usepackage{verbatim}
\usepackage{graphics}
\usepackage{mathpartir}
\usepackage{tikz}
\usepackage{todonotes}

% these two are used to create the wavy division sign
\usepackage{stackengine}
\usepackage{scalerel}

\usepackage{hyperref}
\usepackage[nameinlink, capitalise]{cleveref}

\usepackage{array}


%% Beamer defines a range of its own theorems
\ifx\beamer\undefined

\theoremstyle{plain}
\newtheorem{theorem}{Theorem}[section]
\newtheorem{lemma}[theorem]{Lemma}
\newtheorem{corollary}[theorem]{Corollary}

\theoremstyle{definition}
\newtheorem*{remark}{Remark}
\newtheorem*{notation}{Notation}
\newtheorem{definition}{Definition}
\newtheorem{conjecture}{Conjecture}
\newtheorem{example}{Example}

\else 
  %% Intentionally left blank
\fi

\makeatletter
\newcommand\xlabel[2][]{\phantomsection\def\@currentlabelname{#1}\label{#2}}
\makeatother

\newenvironment{rules}[1][{}]{\begin{mathpar}}{\end{mathpar}}

%% This puts the name on the top
\NewDocumentCommand{\defrule}{o o m m}{
  \inferrule*[lab=\IfNoValueTF{#1}{}{\textsc{[#1]}}]
  { #3 }
  { #4 }
  \IfNoValueTF{#2}{}{\xlabel[#1]{\ifdefined\InApx{apx:}\else\fi#2}}
}


\NewDocumentCommand{\ruleref}{m}{{[\textsc{\nameref{#1}}]}}


\usepackage[abt]{pfpl-syntax}
\usepackage{pfpl-judgments}

%% \xMapsto command
% \usepackage{mathtools}
\usepackage{stmaryrd}

\makeatletter
\newcommand{\xMapsto}[2][]{\ext@arrow 0599{\Mapstofill@}{#1}{#2}}
\def\Mapstofill@{\arrowfill@{\Mapstochar\Relbar}\Relbar\Rightarrow}
\makeatother

%% Instructor-only remarks. These remarks requires the benefit of the hind sight to understand
%% (or foreshadowing) future content so it doesn't make sense to put it in the file
%% Define \isstudentcopy to generate the student version
\definecolor{iremarkcolor}{rgb}{0.0, 0.0, 0.5}
\NewDocumentEnvironment{iremark}{ +b }
{ \ifthenelse{\isundefined{\isstudentcopy}}{
    \begingroup
    \color{iremarkcolor}
    \begin{remark}
    #1
    \end{remark}
    \endgroup
  }{
  }
}{ }

\newcommand{\inlremark}[1]{\ifthenelse{\isundefined{\isstudentcopy}}{\begingroup\color{iremarkcolor}(#1)\endgroup}{}}
\newcommand{\PFPL}{\textbf{\textsf{PFPL}}}
\makeatletter

\NewDocumentCommand{\fstEx}{s O{\tau_1} O{\tau_2} m}{\IfBooleanTF{#1}{\projEx*<1>[#2][#3]{#4}}{\projEx<1>[#2][#3]{#4}}}
\NewDocumentCommand{\sndEx}{s O{\tau_1} O{\tau_2} m}{\IfBooleanTF{#1}{\projEx*<2>[#2][#3]{#4}}{\projEx<2>[#2][#3]{#4}}}

\NewDocumentCommand{\hatM}{}{\hat{M}}

\NewDocumentCommand{\Iff}{}{\,\mathrm{iff}\,}
\NewDocumentCommand{\limp}{}{\supset}

\NewDocumentCommand{\HT}{O{A}}{\mathsf{HT}_{#1}}


\makeatother



\usepackage{newunicodechar}
\newunicodechar{λ}{\ensuremath{\mathnormal\lambda}}
\newunicodechar{→}{\ensuremath{\mathrel\to}}
\newunicodechar{⊢}{\ensuremath{\mathrel\vdash}}
\newunicodechar{∀}{\ensuremath{\mathnormal\forall}}
\newunicodechar{∃}{\ensuremath{\mathnormal\exists}}
\newunicodechar{Γ}{\ensuremath{\mathnormal\Gamma}}

\title{15-791 ATPL \\ Week 8 Notes}
\author{Tesla Zhang, Jason Yao}
\date{\today}

\begin{document}
\maketitle

\section{Introduction}
Last week we introduced the lax framework of computation type.
This week, we introduce another approach to stratify value types and computation types,
called \emph{call-by-push-value} (CBPV), or polarization.

Instead of introducing a type former for computation types,
we maintain the distinction between type structures instead.
We consider positive types \emph{values types}, which we denote as $A^+$,
and negative types \emph{computation types}, which we denote as $A^-$.
Similar to adjoint logic, we link these two families of types with an adjunction.

The syntax of types in CBPV is given by the following grammar:

\section{IO Effects}
We can represent various effects using CBPV as the framework. Here, we represent simple IO with printing to standard out and reading from standard in.


\subsection{Statics}
Assume we have some type $\kw{string}$. We add the following typing rules to CBPV:

\begin{mathpar}
  \defrule[T-Print][sta:print]
  {\Gamma \entails{\isOfTp{s}{\kw{string}}}}
  {\Gamma \entails{\isOfTp{\kw{print}(s)}{\freeTy{\topTy}}}}

  \defrule[T-Read][sta:read]
  {\strut}
  {\Gamma \entails{\isOfTp{\kw{read}}{\freeTy{\kw{string}}}}}
\end{mathpar}

\subsection{Dynamics}

For values, we use the typical evaluation semantics where $V \evalsTo[] V_0$ for some fully evaluated value $V_0$.

For computations, we describe a configuration/state transition system based on a simplified process calculus. 

We have three processes $\cal I$, $\cal O$, and $C$ running in parallel composition, written as 
${\cal I} \parallel {\cal O} \parallel C$. $\cal I$ and $\cal O$ are the input and output processes, respectively, and are represented as lists of \kw{string}s. $C$ is the current running computation.

We also have two channels \kw{stdin} and \kw{stdout} which carry data of type \kw{string}.

\subsubsection{Transitions}
\todo{Needed to use \kw{s} in the step actions instead since $s$ shows up as $\int$ for some reason}
\begin{mathpar}
  \defrule[D-Input][dyn:input]
    {\strut}
    {\consEx*{s}{\cal I} \stepsTo<\sndAc*{\kw{stdin}}{\kw{s}}> {\cal I}}

  \defrule[D-Output][dyn:output]
    {\strut}
    {{\cal O} \stepsTo<\rcvAc*{\kw{stdout}}{\kw{s}}> \consEx*{s}{\cal O}} 

  \defrule[D-Print][dyn:print]
    {\strut}
    {\kw{print}(s) \stepsTo<\sndAc*{\kw{stdout}}{\kw{s}}> {\retEx*{\unitEx*}}}

  \defrule[D-Read][dyn:read]
    {\strut}
    {\kw{read} \stepsTo<\rcvAc*{\kw{stdin}}{\kw{s}}> {\retEx*{s}}}

  \defrule[D-Par][dyn:par]
    {S_1 \stepsTo<\alpha> S_1' \\
     S_2 \stepsTo<\overline{\alpha}> S_2'}
    {S_1 \parallel S_2 \stepsTo S_1' \parallel S_2'} 
\end{mathpar}

\subsection{Equations}

\section{Mutable State}

\subsection{Grammar}
\subsection{Statics}
\subsection{Dynamics}
\subsection{Equations}

\end{document}