\documentclass[letterpaper]{article}

\usepackage{ifthen}
\usepackage{xparse}
\usepackage{enumitem}
\usepackage[utf8]{inputenc}
\usepackage{amsmath}
\usepackage{amssymb}
\usepackage{stmaryrd}
\usepackage{amsthm}
\usepackage{mathtools}
\usepackage{proof}
\usepackage{colonequals}
\usepackage{comment}
\usepackage{textcomp}
\usepackage[us]{optional}
\usepackage{color}
\usepackage{url}
\usepackage{verbatim}
\usepackage{graphics}
\usepackage{mathpartir}
\usepackage{tikz}
\usepackage{todonotes}

% these two are used to create the wavy division sign
\usepackage{stackengine}
\usepackage{scalerel}

\usepackage{hyperref}
\usepackage[nameinlink, capitalise]{cleveref}

\usepackage{array}


%% Beamer defines a range of its own theorems
\ifx\beamer\undefined

\theoremstyle{plain}
\newtheorem{theorem}{Theorem}[section]
\newtheorem{lemma}[theorem]{Lemma}
\newtheorem{corollary}[theorem]{Corollary}

\theoremstyle{definition}
\newtheorem*{remark}{Remark}
\newtheorem*{notation}{Notation}
\newtheorem{definition}{Definition}
\newtheorem{conjecture}{Conjecture}
\newtheorem{example}{Example}

\else 
  %% Intentionally left blank
\fi

\makeatletter
\newcommand\xlabel[2][]{\phantomsection\def\@currentlabelname{#1}\label{#2}}
\makeatother

\newenvironment{rules}[1][{}]{\begin{mathpar}}{\end{mathpar}}

%% This puts the name on the top
\NewDocumentCommand{\defrule}{o o m m}{
  \inferrule*[lab=\IfNoValueTF{#1}{}{\textsc{[#1]}}]
  { #3 }
  { #4 }
  \IfNoValueTF{#2}{}{\xlabel[#1]{\ifdefined\InApx{apx:}\else\fi#2}}
}


\NewDocumentCommand{\ruleref}{m}{{[\textsc{\nameref{#1}}]}}


\usepackage[abt]{pfpl-syntax}
\usepackage{pfpl-judgments}

%% \xMapsto command
% \usepackage{mathtools}
\usepackage{stmaryrd}

\makeatletter
\newcommand{\xMapsto}[2][]{\ext@arrow 0599{\Mapstofill@}{#1}{#2}}
\def\Mapstofill@{\arrowfill@{\Mapstochar\Relbar}\Relbar\Rightarrow}
\makeatother

%% Instructor-only remarks. These remarks requires the benefit of the hind sight to understand
%% (or foreshadowing) future content so it doesn't make sense to put it in the file
%% Define \isstudentcopy to generate the student version
\definecolor{iremarkcolor}{rgb}{0.0, 0.0, 0.5}
\NewDocumentEnvironment{iremark}{ +b }
{ \ifthenelse{\isundefined{\isstudentcopy}}{
    \begingroup
    \color{iremarkcolor}
    \begin{remark}
    #1
    \end{remark}
    \endgroup
  }{
  }
}{ }

\newcommand{\inlremark}[1]{\ifthenelse{\isundefined{\isstudentcopy}}{\begingroup\color{iremarkcolor}(#1)\endgroup}{}}
\newcommand{\PFPL}{\textbf{\textsf{PFPL}}}
\makeatletter

\NewDocumentCommand{\fstEx}{s O{\tau_1} O{\tau_2} m}{\IfBooleanTF{#1}{\projEx*<1>[#2][#3]{#4}}{\projEx<1>[#2][#3]{#4}}}
\NewDocumentCommand{\sndEx}{s O{\tau_1} O{\tau_2} m}{\IfBooleanTF{#1}{\projEx*<2>[#2][#3]{#4}}{\projEx<2>[#2][#3]{#4}}}

\NewDocumentCommand{\hatM}{}{\hat{M}}

\NewDocumentCommand{\Iff}{}{\,\mathrm{iff}\,}
\NewDocumentCommand{\limp}{}{\supset}

\NewDocumentCommand{\HT}{O{A}}{\mathsf{HT}_{#1}}


\makeatother


% \RequirePackage{fourier}
% \RequirePackage{cochineal}
% \RequirePackage{lettrine}

\usepackage{newunicodechar}
\newunicodechar{∀}{\ensuremath{\mathnormal\forall}}
\newunicodechar{∃}{\ensuremath{\mathnormal\exists}}
\newunicodechar{Γ}{\ensuremath{\mathnormal\Gamma}}
\newunicodechar{∈}{\ensuremath{\mathrel\in}}
\newunicodechar{▷}{\ensuremath{\mathrel\vartriangleright}}
\newunicodechar{◁}{\ensuremath{\mathrel\vartriangleleft}}
\newunicodechar{≝}{\ensuremath{\mathrel{:=}}}
\newunicodechar{⇝}{\ensuremath{\mathrel\rightsquigarrow}}
\newunicodechar{⇓}{\ensuremath{\mathrel\Downarrow}}
\newunicodechar{≈}{\ensuremath{\mathrel\approx}}
\newunicodechar{γ}{\ensuremath{\mathnormal\gamma}}
\newunicodechar{δ}{\ensuremath{\mathnormal\delta}}

\NewDocumentCommand{\memState}{m m}{\left\{ #1 \parallel #2 \right\}}
\NewDocumentCommand{\mapsVal}{m m}{#1 \hookrightarrow #2}
\NewDocumentCommand{\letvEx}{m m m}{\Abt{\Opn{letv}}(#1;\Abs<#2>{#3})}
\NewDocumentCommand{\eqExact}{m m m}{#1 \doteq #2 \in #3}

\title{15-791 ATPL \\ Week 8 Notes}
\author{Tesla Zhang, Jason Yao}
\date{\today}

\begin{document}
\maketitle
\tableofcontents

\section{Call-by-Push-Value (CBPV)}
Last week we introduced the lax framework of computation type.
This week, we introduce another approach to stratify value types and computation types,
called \emph{call-by-push-value} (CBPV), or \emph{polarization}.

Instead of introducing a type former for computation types,
we maintain the distinction between type structures instead.
We consider positive types \emph{values types}, which we denote as $A$,
and negative types \emph{computation types}, which we denote as $X$.
The polarity here is related to the polarity of the corresponding connectives in linear logic.
Similar to adjoint logic, we link these two families of types with an adjunction,
where the computation fragment is similar to the linear fragment in adjoint logic.

\subsection{Syntax}

The syntax of types in CBPV is given by the following grammar:

\[
  \begin{array}{rcl}
    \text{Computation types} \quad & X ::= & \prodTy*{X_1}{X_2} \mid \parrTy*{A_1}{X_2} \mid \freeTy*{A} \\
    \text{Value types} \quad       & A ::= & \topTy* \mid \tensorTy*{A_1}{A_2} \mid \sumTy*{A_1}{A_2} \mid \thunkTy*{X} \\
  \end{array}
\]

For simplicity, we introduce the function type only as a computation type,
and for products, we introduce both versions, corresponding to the nature of the product that
it can be used positively or negatively.

The syntax for terms is given by the following grammar:

\[
  \begin{array}{rcl}
    \text{Values} \quad M, V ::=
      && x \mid \unitEx* \mid \tensorEx*{M_1}{M_2} \mid \injEx*<1>{M} \mid \injEx*<2>{M} \mid \suspEx*{C} \\
    \text{Computations} \quad C ::=
      && \retEx*{M} \mid \bndEx*{C_1}{x}{C_2} \mid \pairEx*{C_1}{C_2} \mid \projEx*<1>{C} \mid \projEx*<2>{C} \mid \\
      && \lamEx*{x}{C} \mid \appEx*{C}{M} \mid \forceEx*{M} \mid \checkEx*{M}{C} \mid \\
      && \splitEx*{M}{x_1}{x_2}{C} \mid \caseEx* M x C
  \end{array}
\]

We tend to use $V$ to indicate that a value is, a value, as opposed to some potential extensions of the syntax
where there can be reducible expressions that are not directly values, which we will use $M$.
The syntax above does not allow non-values in the syntax category of values, but we want to work with
an eye on the future.

For positive products $\tensorTy*{A_1}{A_2}$, we overload the tensor operator for both the type formation
and the value introduction due to the presence of another product type taking up the pairing notation.

\subsection{Statics}

The static semantics of CBPV is defined by the following judgment schemas:

\begin{align*}
  &Γ \entails{\isOfTp M A} && \quad \text{Value typing} \\
  &Γ \entails{\isOfTp C X} && \quad \text{Computation typing}
\end{align*}

Similar to the lax framework, the judgment form is stratified into two categories,
but the context used is always a context of value types.
Therefore, we only need to (and only intend to) consider value substitutions
into both of these judgment forms.

The rules for values are given by the typing rules in~\cref{fig:cbpv-values},
which only includes identity and the introduction of value types.

\begin{figure}[ht!]
\centering
\begin{mathpar}
  \defrule[Var][sta:var]
    { \strut }
    { Γ, x:A \entails{\isOfTp x A} }

  \defrule[Unit-I][sta:unit-i]
    { \strut }
    { Γ \entails{\isOfTp{\unitEx*}{\topTy*}} }

  \defrule[Tensor-I][sta:tensor-i]
    { Γ \entails{\isOfTp {M_1} {A_1}} \\ Γ \entails{\isOfTp {M_2} {A_2}} }
    { Γ \entails{\isOfTp{\tensorEx*{M_1}{M_2}}{\tensorTy*{A_1}{A_2}}} }

  \defrule[Sum-I1][sta:sum-i1]
    { Γ \entails{\isOfTp M {A_1}} }
    { Γ \entails{\isOfTp{\injEx*<1> M}{\sumTy*{A_1}{A_2}}} }

  \defrule[Sum-I2][sta:sum-i2]
    { Γ \entails{\isOfTp M {A_2}} }
    { Γ \entails{\isOfTp{\injEx*<2> M}{\sumTy*{A_1}{A_2}}} }

  \defrule[Thunk-I][sta:thunk-i]
    { Γ \entails{\isOfTp C X} }
    { Γ \entails{\isOfTp{\suspEx* C}{\thunkTy* X}} }
\end{mathpar}
\caption{Statics for values}
\label{fig:cbpv-values}
\end{figure}

The rules for computations are given by the typing rules in~\cref{fig:cbpv-computations},
which includes the elimination of value types and all the rules of computation types.

\begin{figure}[ht!]
\centering
\begin{mathpar}
  \defrule[Thunk-E][sta:thunk-e]
    { Γ \entails{\isOfTp M {\thunkTy* X}} }
    { Γ \entails{\isOfTp{\forceEx* M}{X}} }

  \defrule[Free-I][sta:free-i]
    { Γ \entails{\isOfTp M A} }
    { Γ \entails{\isOfTp{\retEx* M}{\freeTy* A}} }

  \defrule[Free-E][sta:free-e]
    { Γ \entails{\isOfTp {C_1} {\freeTy*{A_1}}} \\
      Γ, x:A_1 \entails{\isOfTp {C_2}{X_2}} }
    { Γ \entails{\isOfTp{\bndEx*{C_1}{x}{C_2}}{X_2}} }

  \defrule[Prod-I][sta:prod-i]
    { Γ \entails{\isOfTp {C_1} {X_1}} \\
      Γ \entails{\isOfTp {C_2} {X_2}} }
    { Γ \entails{\isOfTp{\pairEx*{C_1}{C_2}}{\prodTy*{X_1}{X_2}}} }

  \defrule[Prod-E1][sta:prod-e1]
    { Γ \entails{\isOfTp C {\prodTy*{X_1}{X_2}}} }
    { Γ \entails{\isOfTp{\projEx*<1> C}{X_1}} }

  \defrule[Prod-E2][sta:prod-e2]
    { Γ \entails{\isOfTp C {\prodTy*{X_1}{X_2}}} }
    { Γ \entails{\isOfTp{\projEx*<2> C}{X_2}} }

  \defrule[Parr-I][sta:parr-i]
    { Γ, x:A \entails{\isOfTp C X} }
    { Γ \entails{\isOfTp{\lamEx* x C}{\parrTy* A X}} }

  \defrule[Parr-E][sta:parr-e]
    { Γ \entails{\isOfTp C {\parrTy* A X}} \\
      Γ \entails{\isOfTp M A} }
    { Γ \entails{\isOfTp{\appEx* C M}{X}} }

  \defrule[Unit-E][sta:unit-e]
    { Γ \entails{\isOfTp M \topTy*} \\
      Γ \entails{\isOfTp C X} }
    { Γ \entails{\isOfTp{\checkEx* M C}{X}} }

  \defrule[Tensor-E][sta:tensor-e]
    { Γ \entails{\isOfTp M {\tensorTy*{A_1}{A_2}}} \\
      Γ, x_1:A_1, x_2:A_2 \entails{\isOfTp C X} }
    { Γ \entails{\isOfTp{\splitEx* M {x_1}{x_2} C}{X}} }

  \defrule[Sum-E][sta:sum-e]
    { Γ \entails{\isOfTp M {\sumTy*{A_1}{A_2}}} \\
      Γ, x_1:A_1 \entails{\isOfTp {C_1} X} \\
      Γ, x_2:A_2 \entails{\isOfTp {C_2} X} }
    { Γ \entails{\isOfTp{\caseEx* M x C}{X}} }
\end{mathpar}
\caption{Statics for computations}
\label{fig:cbpv-computations}
\end{figure}

We may further extend the CBPV language to include the following constructions,
which are not present in the current syntax:

\begin{description}
  \item[Effects] The computation types can include all sorts of effects,
    such as non-termination, exceptions, state, etc., we will see some of these later.
  \item[Pure functions] We currently only have functions as computations,
    which is usually true and is essential for adding effects,
    but we may also consider adding pure functions as values,
    whose applications are intended to be values, not computations.
    Then we can even distinguish between pure and impure functions in the type system.
  \item[Dependent type formers] A few weeks later, we will be adding dependent types
    to the CBPV framework, which possibly includes dependent function types and dependent pairs.
\end{description}

\subsection{Dynamics}

We refrain from formally describing the dynamics of CBPV in this note,
because the standard presentation of the dynamics of CBPV uses the CK-Machine,
which is different from our usual presentation of the semantics of type theories
based on reductions.
Instead, we make some informal remarks about it.

The CK-Machine can be viewed as a typed version of K-Machine:
when describing dynamics using the K-Machine, we provide the stepping relation as rules
of the following schemas, which we will refer to as a \emph{configuration}:

\begin{align*}
  &K ▷ C && \text{Running an expression with the stacks $K$} \\
  &K ◁ M && \text{Returning a value to the stacks $K$}
\end{align*}

In the CK-Machine, the configuration is defined using the following schema:
\[ Γ \mid C~X_1~K~X_2 \]
Where:
\begin{itemize}
\item $Γ \entails{\isOfTp C X_1}$ is the computation being evaluated,
\item $K$ is the stack that is similar to the K-Machine (not formally defined yet),
\item $X_2$ is the type of the result of the computation.
\end{itemize}
Using our previous convention, it might be more intuitive to write it like this:
\[ K ▷ (Γ \entails{\isOfTp{C}{X_1}}):X_2 \]
But we will stick to the former notation for consistency with the literature.

The transition of the CK-Machine is defined by the following schema:
\[ Γ \mid C~X_1~K~X_2 \stepsTo{} Γ \mid C'~X_1'~K'~X_2 \]
Note that we fix $Γ$ and $X_2$.
Some examples of transition rule in the CK-Machine,
including the rules for the computation type and the function type, are listed below:

\[
\begin{array}{rllllll}
&Γ \mid &\bndEx*{C_1}{x}{C_2}&X_1&K&X_2\\ \stepsTo{}
&Γ \mid &C_1&\freeTy*{X_3}&(\bndEx*{-}{x}{C_2} :: K)&X_2\\
&Γ \mid &\retEx* M&\freeTy*{X_3}&(\bndEx*{-}{x}{C} :: K)&X_2\\ \stepsTo{}
&Γ \mid &[M/x]C&X_1&K&X_2\\

&Γ \mid &\appEx* C M&X_1&K&X_2 \\ \stepsTo{}
&Γ \mid &C&\parrTy*{A}{X_1}&(M :: K)&X_2 \\
&Γ \mid &\lamEx* x C&\parrTy*{A}{X_1}&(M :: K)&X_2 \\ \stepsTo{}
&Γ \mid &[M/x]C&X_1&K&X_2
\end{array}
\]

The stack $K$ is a stack data structure in the traditional sense that can contain the following elements:
\begin{itemize}
\item A value $M$, which is supposed to be applied to the function being computed,
\item A continuation $\bndEx*{-}{x}{C}$,
\item A tag, usually represented as an element in a finite indexing set, for selection in case of sum types.
  Since our sum type is binary, we can use a boolean value to represent the tag.
\end{itemize}
When the stack is empty and the computation results in a $\retEx* M$,
the machine is said to have terminated and computes to the value $M$.

\subsection{Categorical semantics}
\newcommand{\Ctx}{\mathsf{Ctx}}
\newcommand{\Stk}{\mathsf{Stk}}

Under the standard categorical interpretation of type theories,
we may expect the value fragment of CBPV language to be interpreted as a
bicartesian monoidal category,  i.e. with finite products and coproducts,
which is regarded as a category of contexts of value types, denoted $\Ctx$.
Morphisms in this category correspond to value substitutions,
and we identify value types with contexts of length $1$.

What's more interesting is the interpretation of the computation fragment.
Instead of interpreting the terms directly, we interpret the \emph{stacks} of the CK-Machine.
Recall that stacks are used in the following form of judgment:
\[ Γ \mid C~X_1~K~X_2 \]
It says that $Γ \entails{\isOfTp{C}{X_1}}$, and $K$ takes this
computation of type $X_1$ into a computation of type $X_2$.
This is a more natural choice of \emph{morphisms} between computation types,
so the categorical semantics starts from the stacks, and then we interpret the type and term
formers of the computation fragment in terms of these stacks.

Since the stacks are \emph{within} value contexts, therefore they have to
be \emph{indexed} by $\Ctx$, we denote this category as $\Stk$.
Formally speaking, $\Stk$ is a (strict) $\Ctx$-indexed category,
with all the base change functors being identity on objects
(since CBPV is simply-typed after all).
We refer to this construction as a \emph{locally $\Ctx$-indexed category}.

Here are some standard facts about morphisms in $\Stk$:
\begin{itemize}
\item The identity morphism is the empty stack, which operationally does nothing,
  thus composition with them satisfies the identity laws.
\item A stack that takes $X_1$ to $X_2$ within $Γ$ is a morphism in $\Stk(X_1, X_2)$
  in the fiber over $Γ ∈ \Ctx$.
\item Composition of morphisms correspond to the concatenation of stacks,
  defined by induction in the standard way, similar to the concatenation of singly linked lists in SML.
\end{itemize}

For the remaining details, such as the interpretation of computation types and the adjunction
between $\Ctx$ and $\Stk$, we encourage the readers to refer to the writings of Paul Blain Levy.

\subsection{Compiling lax to CBPV}
Recall that the lax framework is a stratified type theory with a type former for computation types,
using the following judgment schemas:
\begin{align*}
&Γ \entails{\isOfTp M A} && \text{Value typing} \\
&Γ \entails{\retsTp E A} && \text{Computation typing} \\
\end{align*}
We may define an interpretation, or a \emph{compiler}, from the lax framework to CBPV.
We describe the translation using the following syntax:
\begin{enumerate}
\item For value of type $A$ in lax, we define $||A||$ to be its interpretation in CBPV, which is a value type;
\item For computation of type $A$ in lax, we define $|A|$ to be $\freeTy*{||A||}$, which is a computation type;
\end{enumerate}
The translation of types is defined below:
\begin{align*}
  ||\unitTy*|| &≝ \unitTy* \\
  ||\prodTy*{A_1}{A_2}|| &≝ \tensorTy*{||A_1||}{||A_2||} \\
  ||\voidTy*|| &≝ \voidTy* \\
  ||\sumTy*{A_1}{A_2}|| &≝ \sumTy*{||A_1||}{||A_2||} \\
  ||\arrTy*{A_1}{A_2}|| &≝ \parrTy*{||A_1||}{||A_2||} \\
  ||\compTy*{A}|| &≝ \thunkTy*{|A|} \\
  |A| &≝ \freeTy*{||A||}
\end{align*}
We extend the translation to contexts in the evident way.
Then, we may define value interpretation $||M||$ and $|M|$ with the following typing soundness theorem in mind:
\begin{theorem}[Soundness]
~
\begin{enumerate}
\item If in lax $Γ \entails{\isOfTp M A}$, then in CBPV $||Γ|| \entails{\isOfTp{||M||}{||A||}}$;
\item If in lax $Γ \entails{\retsTp E A}$, then in CBPV $||Γ|| \entails{\isOfTp{|E|}{|A|}}$.
\end{enumerate}
\end{theorem}
The translation rule is defined using the following schema:
\begin{align*}
&Γ \entails{\isOfTp M A} ⇝ ||M|| \\ 
&Γ \entails{\retsTp E A} ⇝ |E|
\end{align*}
We omit the full transition rules and the proof of soundness, as they are part of the homework.

It is not enough to only define type-sound translation,
we have to make sure that it is also operationally sound.
To do so, we define a \emph{heterogeneous} exact equality, called \emph{correspondence}, between terms in the two languages:
\begin{definition}[Correspondence]
~
\begin{enumerate}
\item $M ≈ V ∈ \compTy*A$ iff $∃ E, C$ such that $M ⇓ \compTy*E$, $V ⇓ \suspEx*C$, and $E \sim C ∈ A$;
\item $E \sim C ∈ A$ iff $∃ V, W$ such that $E \stepsTo(*) \retEx*V$, $C \stepsTo(*) \retEx*W$, and $V ≈ W ∈ A$.
\end{enumerate}
\end{definition}
Then we extend this to open contexts:
\begin{definition}
~
\begin{enumerate}
\item $Γ \gg M ≈ V ∈ A$ iff $∀γ ≈ δ ∈ Γ$, $\hat{γ}(M) ≈ \hat{δ}(V) ∈ A$;
\item $Γ \gg E \sim C ∈ A$ iff $∀γ ≈ δ ∈ Γ$, $\hat{γ}(E) \sim \hat{δ}(C) ∈ A$.
\end{enumerate}
\end{definition}
And we can state the fundamental theorem usually:
\begin{theorem}[Translation Correctness]
~
\begin{enumerate}
\item If $Γ \entails{\isOfTp M A} ⇝ ||M||$, then $Γ \gg M ≈ ||M|| ∈ A$;
\item If $Γ \entails{\retsTp E A} ⇝ |E|$, then $Γ \gg E \sim |E| ∈ A$.
\end{enumerate}
\end{theorem}
We may prove it using the heterogeneous exact equality as the strengthened induction hypothesis,
and the proof is done in homework.

\section{I/O Effects}
We can represent various effects using CBPV as the underlying framework. Here, we introduce simple I/O effects with printing and reading.

\subsection{Grammar}
We introduce a new base type $\kw{string}$. We also introduce two new commands, $\kw{print}$ and $\kw{read}$, which respectively print a string to standard out and read a string from standard in.

\[
    \begin{array}{rl} 
        A ::= & \cdots \mid \kw{string} \\
        C ::= & \cdots \mid \kw{print}(V) \mid \kw{read} 
    \end{array}
\]

\subsection{Statics}
We add the following typing rules for the new commands. Assume that \kw{string} has some sensible typing rules, we just use it as the type we can print/read.

\begin{mathpar}
  \defrule[T-Print][sta:print]
  {\Gamma \entails{\isOfTp{s}{\kw{string}}}}
  {\Gamma \entails{\isOfTp{\kw{print}(s)}{\freeTy{\topTy}}}}

  \defrule[T-Read][sta:read]
  {\strut}
  {\Gamma \entails{\isOfTp{\kw{read}}{\freeTy{\kw{string}}}}}
\end{mathpar}

\subsection{Dynamics}

For values, we use the typical evaluation semantics where $V \evalsTo V_0$ for some fully evaluated value $V_0$.

For computations, we describe a configuration/state transition system based on a simplified process calculus. We have three processes $\cal I$, $\cal O$, and $C$ running in parallel composition, written as
${\cal I} \parallel {\cal O} \parallel C$. $\cal I$ and $\cal O$ are the input and output processes, respectively, and are represented as lists of \kw{string}s. $C$ is the current running computation.

Note that configurations are unique up to structural congruence, so the processes can be freely reordered/nested/etc. as necessary for the dynamics rules to apply (e.g.  ${\cal I} \parallel {\cal O} \parallel C \equiv {\cal I} \parallel C \parallel {\cal O} \equiv {\cal I} \parallel \left( {\cal O} \parallel C \right)$ ).

Finally, we also have two channels \kw{stdin} and \kw{stdout} which carry data of type \kw{string}.

\subsubsection{Transitions}
We present a subset of the state transitions:
\todo{Needed to use \kw{s} in the step actions instead since $s$ shows up as $\int$ for some reason}
\begin{mathpar}
  \defrule[D-Input][dyn:input]
  {\strut}
  {\consEx*{s}{\cal I} \stepsTo<\sndAc*{\kw{stdin}}{\kw{s}}> {\cal I}}

  \defrule[D-Output][dyn:output]
  {\strut}
  {{\cal O} \stepsTo<\rcvAc*{\kw{stdout}}{\kw{s}}> \consEx*{s}{\cal O}}

  \defrule[D-Print][dyn:print]
  {\strut}
  {\kw{print}(s) \stepsTo<\sndAc*{\kw{stdout}}{\kw{s}}> {\retEx*{\unitEx*}}}

  \defrule[D-Read][dyn:read]
  {\strut}
  {\kw{read} \stepsTo<\rcvAc*{\kw{stdin}}{\kw{s}}> {\retEx*{s}}}

  \defrule[D-Single][dyn:single]
  {S_1 \stepsTo<\alpha> S_1'}
  {S_1 \parallel S_2 \stepsTo<\alpha> S_1' \parallel S_2}

  \defrule[D-Par][dyn:par]
  {S_1 \stepsTo<\alpha> S_1' \\
    S_2 \stepsTo<\overline{\alpha}> S_2'}
  {S_1 \parallel S_2 \stepsTo S_1' \parallel S_2'}

\end{mathpar}
Note that rule \ruleref{dyn:par} relies on the definition of $\overline{\alpha}$ representing the complementary action of $\alpha$. In this case, $\sndAc*{a}{s}$, the action to send $s$ over channel $a$, and $\rcvAc*{a}{s}$, the action to receive $s$ over channel $a$, are complementary to each other.

The dynamics for I/O are relatively straightforward. If ${\cal I}$ is not empty and the program commands a \kw{read}, \ruleref{dyn:par} allows the overall configuration to take a step by pairing \ruleref{dyn:input} and \ruleref{dyn:read}. Similarly, if the program wishes to \kw{print}, \ruleref{dyn:output} and \ruleref{dyn:print} will apply concurrently.

\ruleref{dyn:single} allows the program to take steps not pertaining to I/O according to the base CBPV dynamics. In this case $\alpha$ would be $\epsilon$, the silent action.

\todo{Not sure how much of this to mention/is this extraneous detail? \ruleref{dyn:single} wasn't in the lecture, but I felt it was necessary to mention to show that the rest of the program can still run without needing to send messages}
Note that this rule also allows for asynchronous sends since a process could perform an action $\sndAc*{a}{s}$ without a complementary receive. This is not an issue since we can view channels as a message queue, so any asynchronous send just adds its message to the queue. Reads must still be synchronous since there must be a value available on the queue to use in the rest of the program.

\subsection{Equations}
We present a few select equations:
\begin{mathpar}

  \defrule[E-Print-Fun][eq:print-fun]
  {\Gamma \entails{\isOfTp{s}{\kw{string}}} \\
    \Gamma, \isOfTp{x}{A} \entails{\isOfTp{C}{X}}}
  {\Gamma \entails{\seqCmd*{\kw{print}(s)}{\lamEx*{x}{C} \equiv \lamEx*{x}{\seqCmd*{\kw{print}(s)}{C}}}} : \parrTy*{A}{X}}

  \defrule[E-Read-Fun][eq:read-fun]
  { \Gamma, \isOfTp{s}{\kw{string}}, \isOfTp{x}{A} \entails{\isOfTp{C}{X}}}
  {\Gamma \entails{\fletEx*{\kw{read}}{s}{\lamEx*{x}{C} \equiv \lamEx*{x}{\fletEx*{\kw{read}}{s}{C}}}} : \parrTy*{A}{X}}

  \defrule[E-Print-Bind][eq:print-bind]
  {\Gamma \entails{\isOfTp{s}{\kw{string}}} \\
   \Gamma \entails{\isOfTp{C_1}{\freeTy{A_1}}} \\
   \Gamma, \isOfTp{x}{A_1} \entails{\isOfTp{C_2}{X_2}}}
  {\Gamma \entails{\fletEx*{\left( \seqCmd*{\kw{print}(s)}{C_1} \right)}{x}{C_2} \equiv \seqCmd*{\kw{print}(s)}{\fletEx*{C_1}{x}{C_2}}} : X_2}

  \defrule[E-Read-Bind][eq:print-bind]
  {\Gamma, \isOfTp{s}{\kw{string}} \entails{\isOfTp{C_1}{\freeTy{A_1}}} \\
   \Gamma, \isOfTp{x}{A_1} \entails{\isOfTp{C_2}{X_2}}}
  {\Gamma \entails{\fletEx*{\left( \fletEx*{\kw{read}}{s}{C_1} \right)}{x}{C_2} \equiv \fletEx*{\kw{read}}{s}{\fletEx*{C_1}{x}{C_2}}} : X_2}

\end{mathpar}
Note that rules \ruleref{eq:print-fun} and \ruleref{eq:read-fun} say that performing I/O before a function definition is equivalent to performing I/O within the function definition. This might be surprising, since one could imagine that the RHS would not be guaranteed to perform the I/O if the function is not applied. However, we will soon see how to justify this through exact equality.

\subsection{Exact Equality}

\todo{This definition changed...}
This section presented during lecture has been updated, so we will present a higher-level overview of exact equality.

Exact equality between terms $\eqExact{V}{V'}{A}$ is defined as always by induction on $A$ using evaluation dynamics. Essentially, valuable terms are exactly equal if they evaluate to values whose constituent parts are exactly equal. At the base types, they are equal if they are the same value.

For computations $\eqExact{C}{C'}{X}$, we essentially want their behavior to be the same. If the computations return a value, then we want the values to be exactly equal \emph{and} the I/O behavior should be the same. However, if they result in another computation, we \emph{do not} require the I/O behavior to be the same, allowing for the discrepancy seen before in rules \ruleref{eq:print-fun} and \ruleref{eq:read-fun}. This is since ultimately, we only care about the final result of the computation, so intermediate I/O descrepancies are not important.

\section{Mutable State}
We also consider store effects using CBPV. For simplicity, we represent mutable state with a set of predefined assignables.

\subsection{Grammar}
We add two new commands, \kw{get} and \kw{set}, which respectively read and write to memory.
\[
    \begin{array}{rl} 
        C ::= & \cdots \mid \getCmd<a> \mid \setCmd<a>{V}
    \end{array}
\]
Here, $a$ is an assignable, which is essentially a unique identifier for a memory location.

\subsection{Statics}
We introduce a new context $\Sigma$ for assignables that maps assignables to their types. Since we are only considering a predefined set of assignables, $\Sigma$ is fxed.
\[
  \Sigma = \hasTp{a_1}{A_1}, \ldots,\, \hasTp{a_n}{A_n}
\]
Note that each type $A_i$ must be \emph{ground}. A ground type is essentially any value type $A$ \emph{except} $\thunkTy{X}$. If we allowed $\thunkTy{X}$, we would be able to store computations in memory, which allows for partiality through backpatching. We'll see next week that the definition for exact equality is not well formed if there were no restrictions on the types of assignables.

We add the following typing rules for the new commands:
\begin{mathpar}
  \defrule[T-Get][sta:get]
  {\Sigma \entails{\hasTp{a}{A}} \\
   \left( A\, {\sf ground} \right)}
  {\Gamma \entails[\Sigma]{\isOfTp{\getCmd<a>}{\freeTy{A}}}}

  \defrule[T-Set][sta:set]
  {\Sigma \entails{\hasTp{a}{A}} \\
   \left( A\, {\sf ground} \right) \\
    \Gamma \entails[\Sigma]{\isOfTp{V}{A}}}
  {\Gamma \entails[\Sigma]{\isOfTp{\setCmd<a>{V}}{\freeTy{A}}}}
\end{mathpar}
Note that $\setCmd<a>{V}$ returns the value $V$ that was written to memory, instead of just $\unitEx*$.

\subsection{Dynamics}
For values, we again use the typical evaluation semantics where $V \evalsTo[\Sigma] V_0$ for some fully evaluated value $V_0$.

For computations, we have a state transition system where states $\memState{\mu}{C}$ consist of a memory $\mu$ and the current running computation $C$.

$\mu$ is composed of cells mapping assignables to values:
\[
  \mu = \mapsVal{a_1}{V_1} \parallel \cdots \parallel \mapsVal{a_n}{V_n}
\]
For each cell $1 \le i \le n$, we have $\Sigma \entails{\hasTp{a_i}{A_i}}$ and $\Sigma \entails[\varepsilon]{\isOfTp{V_i}{A_i}}$. Note that the empty assignable context $\varepsilon$ is used to indicate $V_i$ cannot reference any memory.

\subsubsection{Transitions}
We present a subset of the state transitions:

\begin{mathpar}
  \defrule[D-Get][dyn:get]
  {\strut}
  {\memState{\mu \parallel \mapsVal{a}{V}}{\getCmd<a>} \stepsTo[\Sigma, \hasTp{a}{A}] \memState{\mu \parallel \mapsVal{a}{V}}{\retEx*{V}}}

  \defrule[D-Set][dyn:set]
  {V \evalsTo[\Sigma] V_0}
  {\memState{\mu \parallel \mapsVal{a}{\_}}{\setCmd<a>{V}} \stepsTo[\Sigma, \hasTp{a}{A}] \memState{\mu \parallel \mapsVal{a}{V_0}}{\retEx*{V_0}}}
\end{mathpar}

\subsection{Equations}
We have the following equations that allow us to ``normalize'' \kw{set} and \kw{get} commands to a sequence of \kw{get}s followed by a \kw{sets}:

\begin{mathpar}

  \defrule[E-Set-Get-Same][eq:set-get-same]
  {\Sigma \entails{\hasTp{a}{A}} \\
   \Gamma \entails[\Sigma]{\isOfTp{V}{A}}}
  {\Gamma \entails[\Sigma]{\seqCmd*{\setCmd<a>{V}}{\getCmd<a>} \equiv \setCmd<a>{V}} : \freeTy{A}}

  \defrule[E-Set-Get-Diff][eq:set-get-diff]
  {\Sigma \entails{\hasTp{a}{A}} \\
   \Sigma \entails{\hasTp{b}{B}} \\
   \left( a \neq b \right) \\
   \Gamma \entails[\Sigma]{\isOfTp{V}{A}}}
  {\Gamma \entails[\Sigma]{\seqCmd*{\setCmd<a>{V}}{\getCmd<b>} \equiv \fletEx*{\getCmd<b>}{x}{\seqCmd*{\setCmd<a>{V}}}{\retCmd*{x}}} : \freeTy{B}}

  \defrule[E-Set-Set-Same][eq:set-set-same]
  {\Sigma \entails{\hasTp{a}{A}} \\
   \Gamma \entails[\Sigma]{\isOfTp{V}{A}} \\
   \Gamma, \isOfTp{x}{A} \entails[\Sigma]{\isOfTp{V'}{A}}
   }
  {\Gamma \entails[\Sigma]{\fletEx*{\setCmd<a>{V}}{x}{\setCmd<a>{V'}} \equiv \letvEx{V}{x}{\setCmd<a>{V'}}} : \freeTy{A}}

  \defrule[E-Set-Set-Diff][eq:set-set-diff]
  {\Sigma \entails{\hasTp{a}{A}} \\
   \Sigma \entails{\hasTp{b}{B}} \\
   \left( a \neq b \right) \\
   \Gamma \entails[\Sigma]{\isOfTp{V}{A}} \\
   \Gamma, \isOfTp{x}{A} \entails[\Sigma]{\isOfTp{V'}{B}}
   }
  {\Gamma \entails[\Sigma]{\fletEx*{\setCmd<a>{V}}{x}{\setCmd<b>{V'}} \equiv \letvEx{V}{x}{\fletEx*{\setCmd<b>{V'}}{y}{\seqCmd*{\setCmd<a>{V}}{\retCmd*{y}}}}} : \freeTy{B}}

  \defrule[E-Get-Set-Same][eq:get-set-same]
  {\Sigma \entails{\hasTp{a}{A}}}
  {\Gamma \entails[\Sigma]{\fletEx*{\getCmd<a>}{x}{\setCmd<a>{x}} \equiv \getCmd<a>} : \freeTy{A}}

  \defrule[E-Get-Get][eq:get-get]
  {\Sigma \entails{\hasTp{a}{A}} \\
   \Sigma \entails{\hasTp{b}{B}}}
  {\Gamma \entails[\Sigma]{\seqCmd*{\getCmd<a>}{\getCmd<b>} \equiv \getCmd<b>} : \freeTy{B}}

\end{mathpar}
We'll see next week how to justify these equations with exact equality.


\end{document}