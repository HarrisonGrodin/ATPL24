% latexmk -pdf week3

\documentclass[letterpaper]{article}

\usepackage{ifthen}
\usepackage{xparse}
\usepackage{enumitem}
\usepackage[utf8]{inputenc}
\usepackage{amsmath}
\usepackage{amssymb}
\usepackage{stmaryrd}
\usepackage{amsthm}
\usepackage{mathtools}
\usepackage{proof}
\usepackage{colonequals}
\usepackage{comment}
\usepackage{textcomp}
\usepackage[us]{optional}
\usepackage{color}
\usepackage{url}
\usepackage{verbatim}
\usepackage{graphics}
\usepackage{mathpartir}
\usepackage{tikz}
\usepackage{todonotes}

% these two are used to create the wavy division sign
\usepackage{stackengine}
\usepackage{scalerel}

\usepackage{hyperref}
\usepackage[nameinlink, capitalise]{cleveref}

\usepackage{array}


%% Beamer defines a range of its own theorems
\ifx\beamer\undefined

\theoremstyle{plain}
\newtheorem{theorem}{Theorem}[section]
\newtheorem{lemma}[theorem]{Lemma}
\newtheorem{corollary}[theorem]{Corollary}

\theoremstyle{definition}
\newtheorem*{remark}{Remark}
\newtheorem*{notation}{Notation}
\newtheorem{definition}{Definition}
\newtheorem{conjecture}{Conjecture}
\newtheorem{example}{Example}

\else 
  %% Intentionally left blank
\fi

\makeatletter
\newcommand\xlabel[2][]{\phantomsection\def\@currentlabelname{#1}\label{#2}}
\makeatother

\newenvironment{rules}[1][{}]{\begin{mathpar}}{\end{mathpar}}

%% This puts the name on the top
\NewDocumentCommand{\defrule}{o o m m}{
  \inferrule*[lab=\IfNoValueTF{#1}{}{\textsc{[#1]}}]
  { #3 }
  { #4 }
  \IfNoValueTF{#2}{}{\xlabel[#1]{\ifdefined\InApx{apx:}\else\fi#2}}
}


\NewDocumentCommand{\ruleref}{m}{{[\textsc{\nameref{#1}}]}}


\usepackage[abt]{pfpl-syntax}
\usepackage{pfpl-judgments}

%% \xMapsto command
% \usepackage{mathtools}
\usepackage{stmaryrd}

\makeatletter
\newcommand{\xMapsto}[2][]{\ext@arrow 0599{\Mapstofill@}{#1}{#2}}
\def\Mapstofill@{\arrowfill@{\Mapstochar\Relbar}\Relbar\Rightarrow}
\makeatother

%% Instructor-only remarks. These remarks requires the benefit of the hind sight to understand
%% (or foreshadowing) future content so it doesn't make sense to put it in the file
%% Define \isstudentcopy to generate the student version
\definecolor{iremarkcolor}{rgb}{0.0, 0.0, 0.5}
\NewDocumentEnvironment{iremark}{ +b }
{ \ifthenelse{\isundefined{\isstudentcopy}}{
    \begingroup
    \color{iremarkcolor}
    \begin{remark}
    #1
    \end{remark}
    \endgroup
  }{
  }
}{ }

\newcommand{\inlremark}[1]{\ifthenelse{\isundefined{\isstudentcopy}}{\begingroup\color{iremarkcolor}(#1)\endgroup}{}}
\newcommand{\PFPL}{\textbf{\textsf{PFPL}}}
\makeatletter

\NewDocumentCommand{\fstEx}{s O{\tau_1} O{\tau_2} m}{\IfBooleanTF{#1}{\projEx*<1>[#2][#3]{#4}}{\projEx<1>[#2][#3]{#4}}}
\NewDocumentCommand{\sndEx}{s O{\tau_1} O{\tau_2} m}{\IfBooleanTF{#1}{\projEx*<2>[#2][#3]{#4}}{\projEx<2>[#2][#3]{#4}}}

\NewDocumentCommand{\hatM}{}{\hat{M}}

\NewDocumentCommand{\Iff}{}{\,\mathrm{iff}\,}
\NewDocumentCommand{\limp}{}{\supset}

\NewDocumentCommand{\HT}{O{A}}{\mathsf{HT}_{#1}}


\makeatother



\usepackage{amsmath, amssymb}

\usepackage[left=1in, right=1in, top=1in, bottom=1in]{geometry}

\newcommand{\rSuspEx}[2]{\Abt{\Opn{susp}}(\Abs<#1>{#2})}
\newcommand{\rkSuspEx}[3]{\Abt{\Opn{susp}}^{(#1)}(\Abs<#2>{#3})}

\newcommand{\emptyStk}{\varepsilon}
\newcommand{\stkStep}[4]{#1 \triangleright #2 \stepsTo #3 \triangleright #4}
\newcommand{\stkMultiStep}[4]{#1 \triangleright #2 \stepsTo(\ast) #3 \triangleright #4}

\setlength\parindent{0pt}

\NewDocumentCommand{\semEq}{m m m}{#1\mathrel{\doteq}#2 \mathrel{\in} #3}
\NewDocumentCommand{\kleeneEq}{m m}{#1\mathrel{\simeq}#2}

\newcommand{\iffdefn}[2]{#1 \; \text{iff} \; #2 }

\title{15-791 ATPL \\ Week 10 Notes}
\author{Colin McDonald (colinmcd) and Karthik Nukala (kvn)}
\date{Lectures: 3/26, 3/28 \\ Scribed: \today}
\begin{document}
\maketitle
\section{Partiality}

To our base call-by-push-value framework, we introduce recursive/self-referential
suspensions.

That is, $\suspEx*{C}$ in base CBPV is transformed into $\rSuspEx{x}{C}$, with statics given by
\begin{equation}\label{eq:rsusp-stc}
    \defrule
    { \Gamma, \isOfTp{x}{U(X)}\entails{\isOfTp{C}{X}} }
    { \Gamma \entails{\isOfTp{\rSuspEx{x}{C}}{U(X)}} }
  \end{equation}
and corresponding dynamics given by
\begin{equation}\label{eq:rsusp-dyn}
    \forceEx*{\rSuspEx{x}{C}}\stepsTo \Sub{\rSuspEx{x}{C}}{x}{C}
\end{equation}

Notice that we can now trivially write programs that never terminate!

Consider $\omega$ defined as follows:
\[
    \omega \triangleq \forceEx*{\rSuspEx{x}{x}}
\]

Now, see what happens when we step $\omega$:
\begin{align*}
    \omega &\triangleq \forceEx*{\rSuspEx{x}{x}} &\text{defn.} \\
    &\stepsTo \Sub{\rSuspEx{x}{C}}{x}{x} &\text{dynamics (\ref{eq:rsusp-dyn})}\\
    &\stepsTo \omega &\text{subst.}
\end{align*}

\subsubsection{Aside: Total Functions}
The fact that $\omega$ exists initially seems to be a curse but it can be a blessing from a certain point of view.

Consider G\"{o}del's T as presented in PFPL/15-312: base types + $\recEx*{e_0}{x}{e_1}({N})$.

Let's try writing the Euclidean algorithm for calculating the greatest
common divisor (GCD) of two numbers.

In SML, this can be mocked up as follows:
\begin{verbatim}
    fun gcd m 0 = m
      | gcd m n = gcd n (m mod n)
\end{verbatim}
\begin{itemize}
    \item Notably, the recursive measure is not immediately obvious! Establishing termination of GCD requires involved reasoning about \verb|mod|, alongside the fact that we swapped arguments in the recursive call.
    \item In T, we're only given $\recEx*{e_0}{x}{e_1}(N)$, wherein the recursive measure is given by $N$ and via the corresponding dynamics rule, goes down by one in subsequent calls.
\end{itemize}
Writing GCD in T is a good, albeit laborious, exercise (solution\footnote{https://gist.github.com/joom/b5d6f69f32e1e8d9026c8a27f7d8de96})

As soon as we introduce partiality in something like PCF, we can immediately write GCD as follows (using a made-up concrete syntax):

\begin{verbatim}
    fix gcd : nat -> nat -> nat is
       \(m: nat): 
       \(n: nat):
       ifz n {
          z     => m
        | s(n') => gcd n (m mod n)
       }
\end{verbatim}

At this point, this is basically the SML code discussed above.

In some sense, the T code is \textit{required} to exhibit explicit values that bound the recursion (by virtue of having to write an $N$ for $\recEx*{e_0}{x}{e_1}(N)$). This serves as a proof of termination, which leads to the punch line:

\begin{center}
    Total programs must also encode their proof of termination!
\end{center}

For any function whose termination is slightly less obvious (like GCD), the proof of termination becomes much longer, resulting in longer code.

This is a result of the Blum Size Theorem\footnote{https://existentialtype.wordpress.com/2014/03/20/old-neglected-theorems-are-still-theorems/}

\subsection{Equational Theories for Partial Functions}

In the usual pursuit of reasoning about our programs,
we'd like to be able to define an equation of the
following sort:

\[
    \defrule
    {\text{(suitable premises)}}
    {\Gamma \entails{\forceEx*{\rSuspEx{x}{C}} \equiv \isOfTp{\Sub{\rSuspEx{x}{C}}{x}{C}}{X}}}
\]

Justifying these definitions requires developing logical relations for our language, taking care to accommodate our newly-introduced partiality.

\begin{definition}[Exact Equality]
    Consider a few cases of the definition:

    \begin{itemize}
        \item $\freeTy \ansTy$:
        \[
            \iffdefn
            {\semEq{C}{C'}{\freeTy \ansTy}}
            {\kleeneEq{C}{C'}}
        \]
        where $\kleeneEq{\_}{\_}$ is defined as follows:
        \[
            \kleeneEq{C}{C'} \triangleq 
            C \stepsTo(\ast) \freeEx{Y/N} \iff C' \stepsTo(\ast) \freeEx{Y/N}
        \]
        Note: If $C\uparrow$, then $C'\uparrow$, otherwise they both terminate with the same $\isOfTp{Y/N}{\ansTy}$ value. Expressing this notion via casework (terminate or not) implicitly uses LEM, so we express it as an $\iff$.
        \item $\freeTy A$:
        \[
            \iffdefn
            {\semEq{C}{C'}{\freeTy A}}
            {\begin{cases}
          C \stepsTo(\ast) \freeEx{M} \iff C' \stepsTo(\ast) \freeEx{M'} \\
          \semEq{M}{M'}{A}
          \end{cases}}
        \]
        That is to say, $C$ and $C'$ both ``co-terminate'' with equal values!
        \item $\parrTy*{A}{X}$: as before
        \item $\thunkTy{X}$:
        \[
            \iffdefn
            {\semEq{M}{M'}{\thunkTy{X}}}
            {\semEq{\forceEx*{M}}{\forceEx*{M'}}{X}}
        \]
    \end{itemize}
\end{definition}

The definition as presented is well-defined. But what's the catch? 

Let's consider how one would prove the reflexivity theorem.
\subsubsection{Reflexivity (Attempt)}

Suppose $\Gamma, \isOfTp{x}{\thunkTy{X}} \gg C \in X$.

WTS: $\Gamma \gg \rSuspEx{x}{C} \in \thunkTy{X}$

Suppose $\gamma \in \Gamma$ (and therefore, $\isOfTp{x}{\thunkTy{X}} \gg \hat{\gamma}C \in X$)

WTS: $\rSuspEx{x}{\hat{\gamma}C} \in \thunkTy{X}$

STS: $\forceEx*{\rSuspEx{x}{\hat{\gamma}C}} \in X$

Via head expansion, STS: $\Sub{\rSuspEx{x}{\hat{\gamma}C}}{x}{C} \in X$

The usual move (of applying the IH to the substitutions) breaks down here: $\rSuspEx{x}{\hat{\gamma}C}$ is exactly what we want to prove!

What's the fix?

\subsubsection{Compactness}

The key observation (common to analyses of partiality) is that
\begin{center}
terminating computations only necessitate \textit{finite} unrollings of the partial computation!
\end{center}

This is akin to our earlier analyses of coinduction: a generator itself may be an infinite object but any computation (given that we were in a total setting at the time of discussion) can only unfold this generator \textit{finitely} many times.

In terms of a while-loop computation, if I have a program 
\[
P \triangleq \texttt{...; while(...)\{\}; ...}
\]
and \textit{I ensure that $P$ terminates}, then $\exists \; n \in \mathbb{N}$ for which
\[
P' \triangleq \texttt{...; int count = 0; while(...)\{...; count++;\}; assert(count == n); ...}
\]
$P'$ as defined above will both not trigger the assert and will be ``equal'' to $P$.

Intuitively, this makes sense but we'd like to prove this.

\begin{itemize}
    \item In a Turing Machine setting, this manifests itself as a Kleene Normal Form\footnote{https://en.wikipedia.org/wiki/Kleene\%27s\_T\_predicate}, which is enough to push the proof through.
    \item However, in a ``cultured'' higher-order setting, things don't seem to work out.
\end{itemize}
Consider an issue that arises when trying to prove the following theorem:

WTS: if $\Sub{\rSuspEx{x}{C_0}}{x}{C} \downarrow \lamEx*{x}{D}$ then $\Sub{\rkSuspEx{k}{x}{C_0}}{x}{C} \downarrow \lamEx*{x}{D}$

where $\rkSuspEx{k}{x}{C_0}$ is defined to roughly mean ``limit the number of recursive calls to $k$ - if more is needed, diverge''.
\\

Are the two $\lambda$'s ($\lamEx*{x}{D}$) on the right side actually equal? No!
\begin{itemize}
    \item $\lamEx*{x}{D}$ in the clause $\Sub{\rSuspEx{x}{C_0}}{x}{C} \downarrow \lamEx*{x}{D}$ has occurrences of $\rSuspEx{x}{C_0}$
    \item $\lamEx*{x}{D}$ in the clause $\Sub{\rkSuspEx{k}{x}{C_0}}{x}{C} \downarrow \lamEx*{x}{D}$ has occurrences of $\rkSuspEx{k}{x}{C_0}$
\end{itemize}

How do we fix this? \underline{The stack machine!}

\[
    \defrule
    {M\evalsTo{\rkSuspEx{0}{x}{C}}}
    {\stkStep{K}{\forceEx*{M}}{K}{\forceEx*{M}}}
    \\ \hspace{0.5in}
    \defrule
    {M\evalsTo{\rkSuspEx{n + 1}{x}{C}}}
    {\stkStep{K}{\forceEx*{M}}{K}{\Sub{\rkSuspEx{n}{x}{C}}{x}{C}}}
\]

Note: in an alternative formulation, we could \emph{define} $\rkSuspEx{n}{x}{C}$ by induction on $n$:
\[
\rkSuspEx{0}{x}{C} \triangleq \forceEx*{\rkSuspEx{0}{x}{C}}
\\ \hspace{0.5in}
\rkSuspEx{n+1}{x}{C} \triangleq \Sub{\rkSuspEx{n}{x}{C}}{x}{C}
\]

Now, we can state compactness as follows:

\begin{theorem}[Compactness]
    If $\stkMultiStep{\Sub{\rSuspEx{x}{C}}{x}{K}}{\Sub{\rSuspEx{x}{C_0}}{x}{c}}{\emptyStk}{\freeEx{V}}$, then for some $n \geq 0$,
    $\stkMultiStep{\Sub{\rkSuspEx{n}{x}{C}}{x}{K}}{\Sub{\rkSuspEx{n}{x}{C_0}}{x}{c}}{\emptyStk}{\freeEx{V}}$, and conversely.
\end{theorem}

\begin{lemma}
    If $x : U(X) \gg \semEq{C}{C'}{X}$ then $\forall n \geq 0, \semEq{\rkSuspEx{n}{x}{C}}{\rkSuspEx{n}{x}{C'}}{U(X)}$
\end{lemma}

\begin{lemma}
    If $x : U(X) \gg \semEq{C}{C'}{X}$ then $\semEq{\rSuspEx{x}{C}}{\rSuspEx{x}{C'}}{U(X)}$
\end{lemma}

\subsubsection{Reflexivity (Full)}

\begin{theorem}[Reflexivity]
    \textnormal{Suppose $\Gamma, x : U(X) \gg C \in X$.}

    \textnormal{WTS: $\Gamma \gg \rSuspEx{x}{C}{\thunkTy{X}}$.}

    \textnormal{Suppose $\gamma \in \Gamma$ (and therefore, $\isOfTp{x}{\thunkTy{X}} \gg \hat{\gamma}C \in X$).}

    \textnormal{WTS: $\rSuspEx{x}{\hat{\gamma}C} \in \thunkTy{X}$.}

    \textnormal{Follows by Lemma 1.3, using the IH with $\gamma$ to get that $\hat{\gamma}(C) \in X$}

    \textnormal{Other cases: as before.}

\end{theorem}

\subsection{Denotational Semantics}

We can define the denotation of $\rSuspEx{x}{C}$ as follows:

\[ \Vert \rSuspEx{x}{C} \Vert = \bigsqcup_{i \geq 0} \Vert \rkSuspEx{i}{x}{C} \Vert \in \Vert U(X) \Vert \]

This definition forms a directed complete partial order (dcpo). However, note that these denotational semantics miss some equivalences that exist in the operational semantics.

This is known as the full abstraction problem (Plotkin).

There is also a further issue: suppose we want to prove some property $\varphi(\rSuspEx{x}{C})$.
It is \underline{insufficient} to show $\forall i \geq 0, \varphi(\rkSuspEx{i}{x}{C})$. For example, consider if $\varphi$ is ``halts in a certain (finite) number of steps'', which holds for any $\rkSuspEx{i}{x}{C}$ but not for $\rSuspEx{x}{C}$ in general.

\section{Introduction to Dependent Types}
(Guided by propositions-as-types)

Per Martin-Löf's Intuitionistic Type Theory has a semantics of intuitionistic logic a la Brouwer, where truth is derived from a fundamental notion of \underline{computation}.

\subsection{Primitive Notions}
\begin{align*}
    A \textbf{ type } & \qquad  A \equiv A' \textbf{ type} \\
    M : A  & \qquad M \equiv M' : A \\[-0.4\baselineskip]
\end{align*}

\vspace{-\baselineskip}
And then we have the notions of \emph{families} of types and elements:
\begin{align*}
    x_1 : A_1, ..., x_n : A_n \vdash A \textbf{ type } & \qquad x_1 : A_1, ..., x_n : A_n \vdash A \equiv A' \textbf{ type} \\
    x_1 : A_1, ..., x_n : A_n \vdash M : A & \qquad x_1 : A_1, ..., x_n : A_n \vdash M \equiv M' : A
\end{align*}

Elementhood (and element equality) is inseparable from typehood (and type equality): there is no phase distinction between programs and specs! In other words, element- and type equality mutually influence each other.

\end{document}
