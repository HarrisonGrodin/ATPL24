\documentclass[letterpaper]{article}
\usepackage{geometry}[margin=1in]


\usepackage{ifthen}
\usepackage{xparse}
\usepackage{enumitem}
\usepackage[utf8]{inputenc}
\usepackage{amsmath}
\usepackage{amssymb}
\usepackage{stmaryrd}
\usepackage{amsthm}
\usepackage{mathtools}
\usepackage{proof}
\usepackage{colonequals}
\usepackage{comment}
\usepackage{textcomp}
\usepackage[us]{optional}
\usepackage{color}
\usepackage{url}
\usepackage{verbatim}
\usepackage{graphics}
\usepackage{mathpartir}
\usepackage{tikz}
\usepackage{todonotes}

% these two are used to create the wavy division sign
\usepackage{stackengine}
\usepackage{scalerel}

\usepackage{hyperref}
\usepackage[nameinlink, capitalise]{cleveref}

\usepackage{array}


%% Beamer defines a range of its own theorems
\ifx\beamer\undefined

\theoremstyle{plain}
\newtheorem{theorem}{Theorem}[section]
\newtheorem{lemma}[theorem]{Lemma}
\newtheorem{corollary}[theorem]{Corollary}

\theoremstyle{definition}
\newtheorem*{remark}{Remark}
\newtheorem*{notation}{Notation}
\newtheorem{definition}{Definition}
\newtheorem{conjecture}{Conjecture}
\newtheorem{example}{Example}

\else 
  %% Intentionally left blank
\fi

\makeatletter
\newcommand\xlabel[2][]{\phantomsection\def\@currentlabelname{#1}\label{#2}}
\makeatother

\newenvironment{rules}[1][{}]{\begin{mathpar}}{\end{mathpar}}

%% This puts the name on the top
\NewDocumentCommand{\defrule}{o o m m}{
  \inferrule*[lab=\IfNoValueTF{#1}{}{\textsc{[#1]}}]
  { #3 }
  { #4 }
  \IfNoValueTF{#2}{}{\xlabel[#1]{\ifdefined\InApx{apx:}\else\fi#2}}
}


\NewDocumentCommand{\ruleref}{m}{{[\textsc{\nameref{#1}}]}}


\usepackage[abt]{pfpl-syntax}
\usepackage{pfpl-judgments}

%% \xMapsto command
% \usepackage{mathtools}
\usepackage{stmaryrd}

\makeatletter
\newcommand{\xMapsto}[2][]{\ext@arrow 0599{\Mapstofill@}{#1}{#2}}
\def\Mapstofill@{\arrowfill@{\Mapstochar\Relbar}\Relbar\Rightarrow}
\makeatother

%% Instructor-only remarks. These remarks requires the benefit of the hind sight to understand
%% (or foreshadowing) future content so it doesn't make sense to put it in the file
%% Define \isstudentcopy to generate the student version
\definecolor{iremarkcolor}{rgb}{0.0, 0.0, 0.5}
\NewDocumentEnvironment{iremark}{ +b }
{ \ifthenelse{\isundefined{\isstudentcopy}}{
    \begingroup
    \color{iremarkcolor}
    \begin{remark}
    #1
    \end{remark}
    \endgroup
  }{
  }
}{ }

\newcommand{\inlremark}[1]{\ifthenelse{\isundefined{\isstudentcopy}}{\begingroup\color{iremarkcolor}(#1)\endgroup}{}}
\newcommand{\PFPL}{\textbf{\textsf{PFPL}}}
\makeatletter

\NewDocumentCommand{\fstEx}{s O{\tau_1} O{\tau_2} m}{\IfBooleanTF{#1}{\projEx*<1>[#2][#3]{#4}}{\projEx<1>[#2][#3]{#4}}}
\NewDocumentCommand{\sndEx}{s O{\tau_1} O{\tau_2} m}{\IfBooleanTF{#1}{\projEx*<2>[#2][#3]{#4}}{\projEx<2>[#2][#3]{#4}}}

\NewDocumentCommand{\hatM}{}{\hat{M}}

\NewDocumentCommand{\Iff}{}{\,\mathrm{iff}\,}
\NewDocumentCommand{\limp}{}{\supset}

\NewDocumentCommand{\HT}{O{A}}{\mathsf{HT}_{#1}}


\makeatother



\NewDocumentCommand{\nonfalseEx}{s}{\Abt{\Opn{nonfalse}}}
\NewDocumentCommand{\HTe}{O{A}}{\widetilde{\mathsf{HT}}_{#1}}
\NewDocumentCommand{\HTk}{O{A}}{\overline{\mathsf{HT}}_{#1}}
\newcommand{\exref}[3]{#1 \gg #2\in #3}
\newcommand{\exrefe}[3]{#1 \gg #2\ \tilde{\in}\ #3}
\newcommand{\exrefk}[2]{#1\ \overline{\in}\ #2}

\title{15-791 ATPL \\ Week 6 Notes}
\author{Mihir Khare and Asher Kornfeld}
\date{\today}
\begin{document}
\maketitle

\section{Introduction}

Previously, we have studied how Tait's method can be used to prove termination of the simply-typed $\lambda$-calculus (STLC), as well as how to apply the argument to various extensions of the STLC.
Most recently, we studied how Tait's method was insufficient for the extension of the STLC with parametric polymorphism, and explored how Girard's method can help us instead.


Here, we formalize an extension of the STLC with a lax formulation of control effects (where the execution of a program transfers control to another part of the program).
We specify the language using stack machine dynamics, and allow the program state to be made explicit in the program with first-class continuations.
We also distinguish pure expressions from impure computations (that potentially have a control effect), and connect the two forms with the \textit{lax modality}.
Finally, we extend Tait's method to prove termination of this extended STLC.

\section{Specification}

\subsection{Syntax}

We first define the syntax for the SLTC extended with continuations and lax modal separation:

$$
  A \coloncolonequals \ansTy* \mid \unitTy \mid \prodTy*{A_1}{A_2} \mid \arrTy*{A_1}{A_2} \mid \contTy{A} \mid \compTy{A}
$$
$$
  M \coloncolonequals x \mid \yesEx \mid \noEx \mid \unitEx* \mid \pairEx*{M_1}{M_2} \mid \projEx*<1>{M} \mid \projEx*<2>{M} \mid \lamEx*{x}{M} \mid \appEx*{M_1}{M_2} \mid \contEx*{K} \mid \compEx{E}
$$
$$
  E \coloncolonequals \retEx*{M} \mid \bndEx*{M}{x}{E} \mid \letccEx*{x}{E} \mid \throwEx*{M_1}{M_2}
$$
$$
  K \coloncolonequals \empStk* \mid \extStk*{K}{x\ .\ E}
$$

Here, $A$ represents types.
The new extensions are $\contTy{A}$ (continuations that refute type $A$) and $\compTy{A}$ (computations of type $A$).

$M$ represents pure expressions mostly as usual.
The new extensions are $\contEx{K}$ (reifying a stack into a value) and $\compEx{E}$ (representing a suspended computation as a value).

$E$ represents impure computation expressions.
It includes methods for creating computations from values (through $\retEx*{M}$), and also for evaluating computation suspensions and binding into a scope (through $\bndEx*{M}{x}{E}$).
It also provides primitives for grabbing the current stack as a value (through $\letccEx*{x}{E}$), and throwing values to a stack (through $\throwEx*{M_1}{M_2}$).

Finally, $K$ represents control flow stacks.
They can either be the final (empty) stack ($\empStk*$), which accepts a value of type $\ansTy$, or a composition of "frames" that accepts some other type ($\extStk*{K}{x\ .\ E}$).

\subsection{Statics}

Aside from the typical pure expression typing judgement, we introduce 2 new typing judgements: $\Gamma \entails{\retsTp{E}{A}}$ for computation expressions, and $\accsTp{K}{A}$ for stacks.
These judgements have the following additional statics rules:

\begin{mathpar}
  \defrule[Lax-I][sta:comp]
    {\Gamma \entails{\retsTp{E}{A}}}
    {\Gamma \entails{\isOfTp{\compEx{E}}{\compTy{A}}}}

  \defrule[Ret][sta:ret]
    {\Gamma \entails{\isOfTp{M}{A}}}
    {\Gamma \entails{\retsTp{\retEx*{M}}{A}}}

  \defrule[Lax-E][sta:bnd]
    {\Gamma \entails{\isOfTp{M}{\compTy{A}}} \\
     \Gamma, \isOfTp{x}{A} \entails{\retsTp{E}{B}}}
    {\Gamma \entails{\retsTp{\bndEx*{M}{x}{E}}{B}}}

  \defrule[Cont-I][sta:letcc]
    {\Gamma, \isOfTp{x}{\contTy{A}} \entails{\retsTp{E}{A}}}
    {\Gamma \entails{\retsTp{\letccEx*{x}{E}}{A}}}

  \defrule[Cont-E][sta:throw]
    {\Gamma \entails{\isOfTp{M_1}{\contTy{A}}} \\
     \Gamma \entails{\isOfTp{M_2}{A}}}
    {\Gamma \entails{\retsTp{\throwEx*{M_1}{M_2}}{B}}}

  \defrule[Emp][sta:emp]
    {}
    {\Gamma \entails{\accsTp{\empStk*}{\ansTy}}}

  \defrule[Frame][sta:frame]
    {\isOfTp{x}{A} \entails{\retsTp{E}{B}} \\
    \accsTp{K}{B}}
    {\accsTp{\extStk*{K}{x\ .\ E}}{A}}

  \defrule[Frame][sta:ret]
    {\accsTp{K}{A}}
    {\Gamma \entails{\isOfTp{\contEx*{K}}{\contTy{A}}}}
\end{mathpar}

\subsection{Dynamics}

The dynamics for much of the language is as we would expect; here, we present some of the new judgements and semantics.
Specifically, with control flow stacks, we have the judgements $K \vartriangleright E$ (computation $E$ is evaluated under context $K$) and $K \vartriangleleft V$ (value $V$ is returned to stack $K$).
These judgements have additional statics rules as follows:

\begin{mathpar}
  \defrule[Eval-OK][sta:eval]
    {\accsTp{K}{A} \\
     \retsTp{E}{A}}
    {\okSt{K \vartriangleright E}}

  \defrule[Value-OK][sta:value]
    {\accsTp{K}{A} \\
     \isOfTp{V}{A}}
    {\okSt{K \vartriangleleft V}}
\end{mathpar}

Note that when evaluating an expression in a given context, we cannot have any free variables in that expression.
We further introduce the judgements $\isVal{M}$ stating that a pure expression is a value, $\iniSt{K \vartriangleright E}$ stating that this context and impure computation are the initial program state, and $\finSt{K \vartriangleleft V}$ stating that this context and value are the final program state.
We also have the notion of pure expression evaluation $M \Downarrow V$, which is the same as $M \stepsTo^* V$.
We can now define the interesting dynamics rules:

\begin{mathpar}
  \defrule[Cont-Val][dyn:cont]
    {}
    {\isVal{\contEx*{K}}}

  \defrule[Comp-Val][dyn:comp]
    {}
    {\isVal{\compEx{E}}}

  \defrule[Ret][dyn:ret]
    {M \Downarrow V}
    {K \vartriangleright \retEx*{M} \stepsTo K \vartriangleleft V}

  \defrule[Bnd][dyn:bind]
    {M \Downarrow \compEx{E}}
    {K \vartriangleright \bndEx*{M}{x}{E'} \stepsTo \extStk*{K}{x\ .\ E'} \vartriangleright E}

  \defrule[Throw][dyn:throw]
    {M_1 \Downarrow \contEx{K_1}}
    {K \vartriangleright \throwEx*{M_1}{M_2} \stepsTo K_1 \vartriangleright \retEx{M_2}}

  \defrule[Letcc][dyn:letcc]
    {}
    {K \vartriangleright \letccEx*{x}{E} \stepsTo K \vartriangleright \Sub{\contEx*{K}}{x}{E}}

  \defrule[Init][dyn:init]
    {}
    {\iniSt{\empStk* \vartriangleright E}}

  \defrule[Final][dyn:final]
    {\isVal{V}}
    {\finSt{\empStk* \vartriangleleft V}}

  \defrule[Shrink][dyn:shrink]
    {\isVal{V}}
    {\extStk*{K}{x\ .\ E} \vartriangleleft V \stepsTo K \vartriangleright \Sub{V}{x}{E}}
\end{mathpar}

\section{Linking to Classical Logic}

The addition of continuations and computations to the STLC allow us to represent classical logic constructively, using "effectful proofs" as our representation.
Specifically, we can treat a pure expression $\isOfTp{M}{A}$ as being a "direct proof" of predicate $A$, and an impure expression $\retsTp{E}{A}$ as being an "indirect proof" of type $A$.
This lets us derive the logical correspondences of $\isOfTp{M}{A} \Leftrightarrow A\ \trueEx$ for pure expressions, $\retsTp{E}{A} \Leftrightarrow A\ \nonfalseEx$ for impure computations, and $\accsTp{K}{A} \Leftrightarrow A\ \falseEx$ for continuations.
Therefore, our variable contexts can be treated as a set of predicates we know are true (along with proofs of them).
We also take the translation of continuation types as $\contTy{A} \Leftrightarrow \neg A$.

\section{Termination}

In order to show that all programs of this extended STLC terminate (and relatedly, show that we can't prove any arbitrary proposition with a non-terminating program), we again turn to Tait's method.
We define the following notions of hereditary termination:
\begin{itemize}
  \item For pure expressions, $\HT[A](M)$ denotes hereditary termination of $M$ of type $A$
  \item For impure computations, $\HTe[A](M)$ denotes hereditary termination of $E$ returning type $A$
  \item For continuations, $\HTk[A](K)$ denotes hereditary termination of $K$ accepting type $A$
\end{itemize}

Their definitions at each type are given as follows:

\begin{itemize}
  \item $\HT[\ansTy](M)$ iff $M \Downarrow \yesEx$ or $M \Downarrow \noEx$
  \item $\HT[\unitTy](M)$ iff $M \Downarrow \unitEx*$
  \item $\HT[\prodTy*{A_1}{A_2}](M)$ iff $M \Downarrow \pairEx*{M_1}{M_2}$ where $\HT[A_1](M_1)$ and $\HT[A_2](M_2)$
  \item $\HT[\arrTy*{A_1}{A_2}](M)$ iff $M \Downarrow \lamEx*{x}{M_2}$ and if $\HT[A_1](M_1)$, then $\HT[A_2](\Sub{M_1}{x}{M_2})$
  \item $\HT[\contTy{A}](M)$ iff $M \Downarrow \contEx*{K}$ where $\HTk[A](K)$
  \item $\HT[\compTy{A}](M)$ iff $M \Downarrow \compEx{E}$ where $\HTe[A](E)$
  \item $\HTk[A](K)$ iff $\HT[A](V)$ implies $K \vartriangleleft V \Downarrow$
  \item $\HTe[A](E)$ iff $\HTk[A](K)$ implies $K \vartriangleright E \Downarrow$
\end{itemize}

The termination judgements for pure expressions are straightforward; for expressions of base types, they must evaluate to the base values, and for expressions of higher types, they must evaluate to "well-behaved" values of that type.
For continuation and computation termination, we have a slightly more interesting formulation.
Termination on stacks is defined in terms of eliminations; stacks are well-behaved if and only if upon returning any well-behaved value to it, evaluation will eventually terminate.
And termination on impure computations is defined in terms of introductions; they are well-behaved if and only if evaluating them in any well-behaved context will eventually terminate.
This flows fairly naturally from how computations and stacks work intuitively; stacks are only useful if you can return values to them, and computations are only useful if you can evaluate them in some context.

We also define the judgements $\exref{\Gamma}{M}{A}$ to mean that if $\HT[\Gamma](\gamma)$ then $\HT[A](\hat{\gamma}(M))$, $\exrefe{\Gamma}{E}{A}$ to mean that if $\HT[\Gamma](\gamma)$ then $\HTe[A](\hat{\gamma}(E))$, and finally $\exrefk{K}{A}$ to mean that $\HTk[A](K)$.
Note that the continuation judgement does not involve a context.
Finally, we present the relevant lemmas and the fundamental theorem:

\begin{lemma}[Head Expansion]
  If $\HT[A](M)$ and $M' \stepsTo M$, then $\HT[A](M')$.
\end{lemma}

\begin{lemma}[Expression Termination]
  If $\HT[A](M)$, then $M \Downarrow$.
\end{lemma}

\begin{theorem}[Fundamental Theorem]
  All language constructs are computable at their respective type:
  \begin{itemize}
    \item If $\Gamma \entails{\isOfTp{M}{A}}$, then $\exref{\Gamma}{M}{A}$.
    \item If $\Gamma \entails{\retsTp{E}{A}}$, then $\exrefe{\Gamma}{E}{A}$.
    \item If $\accsTp{K}{A}$, then $\exrefk{K}{A}$.
  \end{itemize}
\end{theorem}

This is proved by induction on the typing judgements.
A particular corollary of the fundamental theorem is that if we have some closed computation $\empCtx \entails{\retsTp{E}{\ansTy}}$, then $\empStk* \vartriangleright E$ will terminate.
In other words, any well-typed initial-state program is guaranteed to terminate!

\section{Equational Laws}

We also have an extensive notion of reduction/equational laws that apply to the extended STLC.
These rely on our semantic understanding of how the control stack changes within structures like $\bndEx*{M}{x}{E}$ and $\letccEx*{x}{E}$ to define equivalences with different structures.
As before, we also define a notion of exact equality to prove that these rules are correct.

\end{document}
